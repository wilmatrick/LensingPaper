\documentclass[10pt,a4paper]{article}
\usepackage[utf8]{inputenc}
\usepackage{amsmath}
\usepackage{amsfonts}
\usepackage{amssymb}
\usepackage{hyperref}	%for URLs
\usepackage[usenames,dvipsnames]{xcolor}	%for color text
\newcommand{\Comment}[1]{\textsl{\textcolor{Blue}{$\Longrightarrow$ {#1}}}}
\newcommand{\RM}{{\sl RoadMapping}}


\usepackage[usenames,dvipsnames]{xcolor}
\newcommand{\Wilma}[1]{\textcolor{Magenta}{#1}}
\newcommand{\Glenn}[1]{\textcolor{Green}{#1}}
\newcommand{\Aaron}[1]{\textcolor{Orange}{#1}}

\begin{document}
\begin{flushright}
Wilma H. Trick\\
Max-Planck-Institut f\"{u}r Astronomie\\
K\"{o}nigstuhl 17\\
69117 Heidelberg, Germany
\end{flushright}

\noindent Anna Evripidou\\
Assistant Editor\\
"Monthly Notices" and "Geophysical Journal International"\\
Royal Astronomical Society\\

\begin{flushright}
Heidelberg, 4th of July 2016
\end{flushright}

\textbf{Subject:} Answer to the Referee Report on "A spiral galaxy's mass distribution uncovered through lensing and dynamics"\\\\

Dear Ms. Evripidou,

Dear anonymous referee,\\

thank you very much for having a detailed look at our paper draft and preparing this helpful report. We made an effort to account for all the referee's comments and to make some of the sections a bit more understandable---we hope you agree. We attach the original letter and report below to which we added our answers to the referee's comments in blue and italics. \\\\
Kind regards,\\
Wilma Trick\\\\\\

\noindent\textbf{Attachments:}
\begin{itemize}
\item \texttt{Trick{\_}J1331{\_}Modelling{\_}revised.tex/.pdf} - revised paper file ready for publishing (without any marked text or annotations), if the referee agrees.
\item \texttt{Trick{\_}J1331{\_}Modelling{\_}revised{\_}all{\_}changes.pdf} - file in which all changes with respect to the previous version of the paper are marked (new text = green, deleted text = red, annotations = blue).
\item \texttt{Trick{\_}J1331{\_}Modelling{\_}revised{\_}for{\_}referee.pdf} - file for the referee with the deleted text excluded to be more readable (new text = green, annotations = blue).
\end{itemize}



\newpage


\noindent From:  	ae@ras.org.uk\\
Subject:  	MNRAS: MN-16-0473-MJ\\
Date:  	Mon, March 14, 2016 1:00 pm\\
To:  	trick@mpia.de\\
Cc:  	trick@mpia.de,glenn@mpia.de,dutton@nyu.edu\\\\

Dear Ms Trick\\

Copied below are the reviewer's comments on your manuscript entitled "A spiral
galaxy's mass distribution uncovered through lensing and dynamics", ref.
MN-16-0473-MJ, which you submitted to Monthly Notices of the Royal Astronomical
Society.\\

Minor revision of your manuscript is requested before it is reconsidered for
publication.\\

You should submit your revised version, together with your response to the
reviewer's comments via the Monthly Notices ScholarOne Manuscripts site
https://mc.manuscriptcentral.com/mnras.
Enter your Author Centre, where you will find your manuscript title listed under
"Manuscripts with Decisions." Under "Actions," click on "Create a Revision." Your
manuscript reference will be appended to denote a revision.\\

IMPORTANT: do not submit your revised manuscript as a new paper!\\

You will not be able to make your revisions to the originally submitted files of the
manuscript held on ScholarOne Manuscripts. Instead, you must delete the original
files and abstract and replace them with your revised files. Check that any requests
for colour publication or online-only publication are correct. Carefully proof read
the resulting PDF and HTML files that are generated. If you have used a .bib file to
generate your bibliography in Latex, please include this in your .tar archive along
with the .bbl and .tex files; this will aid the editing and typesetting process.\\

When submitting your revised manuscript, you will be able to respond to the comments
made by the reviewer in the space provided. You should also use this space to
document any changes you make to the original manuscript. In order to expedite the
processing of the revised manuscript, please be as specific as possible in your
response to the reviewer. Changes to the manuscript should be highlighted (e.g. in
bold or colour), to assist the referee and editor. Along with the highlighted
manuscript you should also upload clean files (remove any bold font/track changes)
for our publishers, as accepted manuscripts are now immediately published online
ahead of the proof-corrected version.\\

Because we are trying to facilitate timely publication of manuscripts submitted to
MNRAS, your revised manuscript should be uploaded promptly. If you do not submit
your revision within six months, we may consider it withdrawn and request it be
resubmitted as a new submission.\\

Please note that, due to the tight schedule, any post-acceptance changes notified
after the paper has gone into production (i.e. the day after the acceptance email is
sent) cannot be incorporated into the paper before it is typeset. Such changes will
therefore need to be made as part of the proof corrections. To avoid excessive proof
corrections and the delay that these can cause, you are strongly encouraged to
ensure that each version of your paper submitted to MNRAS is completely ready for
publication!\\

I look forward to receiving your revised manuscript.\\

Regards,\\

Anna\\

\noindent Anna Evripidou\\
Assistant Editor\\
"Monthly Notices" and "Geophysical Journal International"\\
Royal Astronomical Society\\
Email: ae@ras.org.uk\\
Tel: (+44) 0207 734 3307 Ext. 124 (except Wednesday)\\
Tel: (+44) 0207 734 3307 Ext. 117 (Wednesday)\\


This message is confidential and is intended solely for the use of the individual(s)
to whom it is addressed.  If you have received this message in error, do not open
any attachment but please notify the sender (above) and delete this message from
your system.\\

cc: all listed co-authors.\\

\noindent Reviewer's Comments:\\
Reviewer: 1\\

Comments to the Author\\

====================================
\begin{itemize}

\item This is an interesting paper which makes neat use of the combination of stellar
kinematics and strong lensing to investigate the mass structure of the spiral galaxy
J1331. \\\Comment{Thanks. It is a very exiting galaxy that's definitely worth exploring.}

\item The introduction could be re-written slightly to emphasise the points that the study
wants to investigate; for instance, the opening paragraph talks about the methods
used rather than the astrophysics that the study is hoping to probe. This could
probably be removed altogether, and the second paragraph promoted to the beginning. \\\Comment{That is a good point. We changed the opening paragraph to a general sentence why determining the matter distribution of galaxies is important. We mention the different methods now in the third paragraph.}

\item It would also be nice to include some more discussion of the possibility of a recent
merger in the galaxy's past in the introduction, as this is a really interesting
feature of the galaxy in question and is also a very active topic of research. \\\Comment{Good point. 1.) We added a sentence mentioning the important role of mergers in modifying the matter distribution and added some recent references. 2.) We added a short paragraph later in the introduction that points out that J1331 is a special and very interesting galaxy for researching mergers. We also explain better that this merger is one of the reasons why we especially focus on the inner regions of J1331 in our study.}

\item The manuscript in its current form is also slightly longer than necessary at certain
points: for instance, in section 3.2.1, it is not really necessary to present all
the lensing equations (e.g. the time delay, equation 7), and similarly, section
3.3.1 contains a lot of material that is implicit in Cappellari's JAM code, and
therefore does not need to be explained equation by equation here. Section 3.3.4
also rather throws the book at the NFW profile relations – it probably isn't
necessary to include equation 22. \\\Comment{We made an effort to remove unnecessary equations in Section 3.2.1, tighten the explanation of the derivation of the Jeans equations in Section 3.3.1 and 3.3.3 and removed big and clumsy mathematical expressions by referring to the explicit equations in the literature instead. In 3.3.2 we refer stronger to the changes we made to Capellari's JAM code and kept the corresponding equations. In Section 3.3.4 we moved equations that are definitions and therefore needed inline to make it shorter; we kept the concentration vs. halo mass relation (former equation 22) and explicitely mention that we will use this later as a prior in our modelling. In general we include a lot of equations and a substantial theory introduction because it is a paper that is potentially interesting for two communities, the lensing and the dynamics community. We wanted to make the paper understandable for both.}

\item From a scientific perspective, in section 4.3.3, the authors comment that the
varying M/L JAM model is unable to reproduce the dynamics when M/L is forced to be
always rising as a function of radius, but it would be interesting to see if the
model could be improved by removing this constraint. Perhaps the authors could add
such a model to this section, or else comment on why they do not attempt it. \\\Comment{We have indeed tried a ``mass-follows-light'' JAM model with constant anisotropy and free M/L profile early on in our investigations of J1331. This was unsuccessful to a degree that showing the result in the paper would not be reasonable.  The way how we can generate variable M/L profiles with our MGE mass model is restricted. We have 6 data points and 5 fit M/L parameter within 5 arcsec. This lead to an over-fitting in the inner regions where most of the Gaussians are, while not having enough flexibility at larger radii. (We added a sentence mentioning that in the paper.) The resulting profile was very wiggly, difficult to interpret and most likely unphysical.  As we lay out in the discussion, constraints on a varying stellar M/L ratio should be motivated by stellar population analysis in the future.}

\item In section 4.4.1, the NFW profile is fitted only to the inner data, and then is
shown to not provide a good model for the data beyond that point. This is not
entirely surprising, given that the outer data were not used to constrain the model!
What happens if the NFW profile is fitted to both small- and large- radius data at
once? (Perhaps it is impossible to find a good NFW model across the whole radius
range, but if this is the case, then it should be stated.) \\\Comment{Yes, you are completely right. Our initial tests tried of course to fit the whole range of data, but failed. The problem is that we were never able to fit both the central $v_\text{rms}$ dip and the dip around $R'\sim 6''$ with any of the models we tried. If you think this is frustrating, we definitely feel the same way. Our best explanation was, that the spiral arms do weird things to the kinematics around $R'\sim 5-7''$ that cannot be captured by the axisymmtric Jeans models. In this region our MGE models also were not a good fit to the galaxy anymore. Considering also that there are other studies focussing on the outer regions and the counter-rotating core is the most interesting feature, we decided to focus our efforts on the inner regions and try to explain the central kinematics only. We added however a small new section 4.4.4 that discusses that and why modelling attempts to the whole data range fail and what possible next modelling steps should be.}

\item The discussion section is too non-committal and speculative in parts. For instance,
in section 5.1, the M/Ls inferred from a number of the different models are compared
with M/Ls from the literature, but the authors do not make any value judgement
between the different models to come to any firmer conclusions about what they
believe the most convincing description of the IMF to be. At the moment, this
section seems to just be trying to show that their various models can be made to
agree with previous models from the literature, but it would be much more
interesting if some attempt could be made to resolve all the different models (both
in this paper and in previous works) or at least present them in a more coherent
framework, to try to come up with a better picture of the nature of J1331's IMF. \\\Comment{This is a valid point of criticism. We thought again about the different models and indications for different IMFs that we collected and came to the conclusion that an IMF slightly less bottom-beavy than the Salpeter IMF is the most likely model for J1331's bulge. We tried to give Section 5.2 (former 5.1) more structure and tweaked the discussion to present arguments for $\Upsilon_{I,*} \sim 4.2$ and and to argue against the alternative models.\\We also moved the section that discusses J1331's merger history to the beginning to the discussion section, focussed more on what the merger could have done to J1331's mass distribution and less on modelling failures. All following sections use the background information about mergers to discuss our models.\\Our scientific honesty is however, that without further data, we cannot be more commital and some of our statements have to stay in the "speculation regime".}

\item In Section 5.2, the authors could make suggestions for what might be causing the two
$v_{rms}$ dips. Currently, they just state what cannot be the cause, but do not make any
more convincing suggestions – e.g. could it be associated with the possible recent
merger that is explored in Section 5.3? It would be more satisfying if some
suggestions like this could be put forward. \\\Comment{Good point. We admit that the text was formulated a bit too negative, and that we can do better in explaining J1331 based on our models than just ruling out things. Also, some of the effects that mergers can have on galaxies \emph{could}, if they occured in J1331, actually explain some of the peculiar kinematic features of J1331. We tried to re-formulate section 5.3 (former 5.2) and discuss several aspects of the kinematics with respect to the merger history.}

\item While the use of English is impressive given that it is not the main author's mother
tongue, I would also recommend giving the manuscript to a native speaker to correct
the number of grammatical errors that remain. \\\Comment{One of the authors is a native speaker and he had a look at the language again. We corrected everything he thought should be corrected. Hopefully it is fine now.}

\end{itemize}


\end{document}
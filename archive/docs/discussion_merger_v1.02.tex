\subsection{On J1331's possible merger history}

\paragraph{Counter-rotating core and minor merger.} J1331 has a large counter-rotating stellar core within $\sim 2$ arcsec. This suggests a process in J1331's past in which two components with angular momenta oriented in opposite directions were involved. Accretion of gas on retrograde orbits and subsequent star formation could lead to a younger and counter-rotating stellar population [TO DO: REF]. However, the to form enough stars such that the net rotation of the large and massive core is retrograde, a very large amount of gas would have had to be accreted by J1331- which is not very likely [TO DO: REF]. Another scenario are galaxy mergers. The products of major mergers of similarly massive disk galaxies are often quiescent, elliptical galaxies \citep{1972ApJ...178..623T} [TO DO: correct reference???]. Therefore a minor merger is more likely to have occured in J1331. The dense nucleus of a satellite galaxy on a retrograde orbit could survive the dissipationless accretion and could spiral to core due to tidal friction \citep{1984ApJ...287..577K}. Usually large bulges of spirals or ellipticals are reddest in the center and become bluer further out.  Because star formation in low-mass satellite galaxies occurred later than in massive galaxies [TO DO: REF], the merger remnant would now host a younger and bluer population with lower velocity dispersion within the older and redder bulge of the massive progenitor. Such a reverse color gradient could correspond to an increasing stellar mass-to-light ratio.
\\The minor merger theory could explain and support our findings from dynamical modelling and the previous sections, that J1331 might have an increasing stellar mass-to-light ratio within its core. 

\paragraph{Possible modelling failures for merger remnants.} Another effect that minor mergers can have on galaxies are a misalignement (warp) of the photometric and kinematic major axis [TO DO: REF]. Would this be the case for J1331 the assumption of axisymmetry is not true anymore and our dynamical modelling would fail.
\\Mergers also could have changed the 3D shape of the dark matter halo and the NFW halo was therefore not a good model for J1331's dark matter halo. We also tried to model J1331 with a cored logarithmic halo. However, due to degeneracies in the modelling, we were not able to either constrain the profile for a cored logarithmic halo, or to rule it out. 


\subsection{Future work}

Standard JAM modelling approaches seem not to work for J1331. A JAM model for J1331 would need to allow for stellar mass-to-light ratio gradients within the galaxy, velocity anisotropy and a dark matter halo. Because of degeneracies between stellar and dark mass, and matter distribution and anisotropy profile, such a dynamical model would not lead to very tight constraints on the model parameters.  A search for color gradients in J1331 and/or investigation of absorption line indices could support or contradict the suspicion of the existence of stellar mass-to-light ratio gradients in J133. Subsequently detailed stellar population analyses of the spectra taken along J1331's major axis should be conducted to constrain the mass-to-light ratio reliably. In addition, the dynamical modelling should use more of the available information on J1331 and fit dynamics (stellar and gas kinematics from \citet{SWELLSV}) simultaneously with the gravitational lensing (image positions, shape and even flux ratios) in a similar fashion to \citet{SWELLSIV}. To also model the extent, shape and flux of the lensing images, the method by \citet{2004ApJ...611..739T,2003ApJ...590..673W} could be employed, which models the surface brightness distribution of the images and source on a pixelated grid. However, for this to work a good model for the galactic extinction would be needed. High-resolution integral-field spectroscopy could help with this, allow for spatially resolved stellar population analysis and dynamical modelling in two dimensions. This would lead to much better understanding of J1331's structure and mass distribution and therefore answer questions on how minor mergers might modify spiral galaxies.
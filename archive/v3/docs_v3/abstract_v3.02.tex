We investigate the matter distribution of a spiral galaxy with counter-rotating stellar core, SDSS J1331+3628 (J1331), independently with gravitational lensing and dynamical modelling. By fitting a gravitational potential model to a quadruplet of lensing images around J1331's bulge, we tightly constrain the mass inside the Einstein radius $R_\text{ein}=(0.91\pm0.02)''(\simeq1.83\pm0.04~\text{kpc})$ to within 4\%: $M_\text{ein} = (7.8\pm0.3) \cdot 10^{10} M_\odot$. We model observed long-slit major axis stellar kinematics in J1331's central regions by finding Multi-Gaussian Expansion (MGE) models for the stellar and dark matter distribution that solve the axisymmetric Jeans equations. The lens and dynamical model are independently derived, but in very good agreement with each other around $\sim R_\text{ein}$. We find that J1331's center requires a steep total mass-to-light ratio gradient. A dynamical model including a NFW halo (with virial velocity $v_{200} \simeq 240 \pm 40~\text{km s}^{-1}$ and concentration of $c_{200} \simeq 8 \pm 2$) and moderate tangential velocity anisotropy ($\beta_z \simeq -0.4 \pm 0.1$) can reproduce the signatures of J1331's counter-rotating core and predict the stellar and gas rotation curve at larger radii. However, our models do not agree with the observed velocity dispersion at large radii. We speculate that the reason could be a non-trivial re-distribution of matter due to a possible merger event in J1331's recent past.

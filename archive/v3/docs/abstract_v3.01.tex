We analyse the stellar and dark matter distribution in the spiral galaxy SDSS J1331+3628 (J1331) by means of two independent methods: gravitational lensing and dynamical modelling. Hubble Space Telescope (HST) imaging by the SWELLS survey shows, that J1331's bulge is superimposed by a quadruplet of extended lensing images. By fitting a gravitational potential model to the image positions, we tightly constrain the mass inside the Einstein radius $R_\text{ein}=(0.91\pm0.02)''(\simeq1.83\pm0.04~\text{kpc})$ to within 4\%: $M_\text{ein} = (7.8\pm0.3) \cdot 10^{10} M_\odot$. According to long-slit major axis stellar kinematics, J1331 has a counter-rotating stellar core inside $\sim 2''(\simeq4.02~\text{kpc})$. We model the observed stellar kinematics in J1331's central regions by finding Multi-Gaussian Expansion (MGE) models for the stellar and dark matter distribution that solve the axisymmetric Jeans equations. We find that J1331 requires a steep total mass-to-light ratio gradient in the center to reproduce the observed stellar kinematics. The best fit dynamical model predicts a total mass inside the Einstein radius consistent with the lens model, and vice versa the lens model gives an successful prediction for the observed kinematics in the galaxy center. For a dynamical model including a NFW dark matter halo,  we constrain the halo to have virial velocity $v_{200} \simeq 240 \pm 40~\text{km s}^{-1}$ and a concentration of $c_{200} \simeq 8 \pm 2$ in case of a moderate tangential velocity anisotropy of $\beta_z \simeq -0.4 \pm 0.1$. The NFW halo models can successfully reproduce the signatures of J1331's counter-rotating stellar core and predict J1331's stellar and gas rotation curve at larger radii. However, all these models are more massive than expected from the stellar velocity dispersion at larger radii alone. We speculate that this could indicate a non-trivial re-distribution of matter due to a possible merger event in J1331's past. This could have resulted in a stellar core in J1331's very center with a Milky-Way-like Kroupa/Chabrier initial mass function (IMF), surrounded by a extended stellar bulge population with an IMF even more bottom-heavy than a Salpeter IMF \Wilma{[TO DO: Word limit is 250.]}.
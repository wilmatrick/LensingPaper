\subsection{Future work}

Standard JAM modelling approaches seem not to work for J1331. A JAM model for J1331 would need to allow for stellar mass-to-light ratio gradients within the galaxy, velocity anisotropy and a dark matter halo. Because of degeneracies between stellar and dark mass, and matter distribution and anisotropy profile, such a dynamical model would not lead to very tight constraints on the model parameters.  A search for colour gradients in J1331 and/or investigation of absorption line indices could support or contradict the suspicion of the existence of stellar mass-to-light ratio gradients in J1331. Subsequently detailed stellar population analyses of the spectra taken along J1331's major axis should be conducted to constrain the mass-to-light ratio reliably.  \Wilma{[TO DO: Include in this discussion the findings from the color profile.]}

In addition, the dynamical modelling should use more of the available information on J1331 and fit dynamics (stellar and gas kinematics from \citet{SWELLSV}) simultaneously with the gravitational lensing (image positions, shape and even flux ratios) in a similar fashion to \citet{SWELLSIV}. To also model the extent, shape and flux of the lensing images, the method by \citet{2004ApJ...611..739T,2003ApJ...590..673W} could be employed, which models the surface brightness distribution of the images and source on a pixelated grid. However, for this to work a good model for the galactic extinction would be needed. 

High-resolution integral-field spectroscopy could help with this, allow for spatially resolved stellar population analysis and dynamical modelling in two dimensions. \Wilma{[TO DO: Include in this discussion that this could also help test, if the vrms dip is maybe because of twist, i.e., the measurements where not done along the major axis in the disk]} This would lead to much better understanding of J1331's structure and mass distribution and therefore answer questions on how minor mergers might modify spiral galaxies.

\Wilma{[TO DO: Include in discussion: High resolution IFU: high spatial resolution --> to resolve small features. High spectral resolution --> because the velocity dispersion in the outer regions of the galaxy are very low and to measure them accurately, need high spectral resolution.]}



\Wilma{[TO DO: Include the following discssion somwhere: The above discussion motivates the following speculation: In the absence of a strong DM contribution in the center, the overall stellar-mass-to-light ratio within the Einstein radius indicates a bottom-heavy IMF close to the Salpeter IMF, consistent with estimates from the velocity dispersion. At the same time the central dip can then be only explained, if there was an increase in stellar mass-to-light ratio with radius \textit{within} the bulge. \emph{If} there was a central stellar population with an IMF close to the \citet{Chabrier2003} IMF, surrounded by a more bottom-heavy population, the central stellar kinematics would be well explained and be fully consistent with the lensing results. (According to Figure \ref{fig:lenscompareboth} the lensing result might not predict a mass to light ratio gradient inside $R_\text{ein}$, but then again the mass slope $alpha=1$ was only weakly constrained.) The disk of J1331 has a lower $\Upsilon_\text{I,*}$ than the bulge (see Table \ref{tab:previousresults} according to \citet{SWELLSI}). Such a drop in $\Upsilon_\text{I,*}$ at the transition region from bulge to disk around $\sim 5''$ could lead to the observed drop in $v_\text{rms}$, while an increasing contribution of a lower mass DM halo at larger radii as found by \citet{SWELLSV}, would recover the kinematics in the out regions of J1331's disk.]}

\Wilma{[TO DO: Why are disk and bulge mass in SWELLS I different to the one in SWELLS IV??]}


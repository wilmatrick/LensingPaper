\subsection{Fitting the Lens Model to the Image Positions}

\paragraph{Image Positions.} We determine the positions of the lensing images by first subtracting a smooth model for the galaxy's surface brightness from the original image. As models we use MGE fits (cf. \S\ref{kap:MGE}) and IRAF ellipse fits (???) to the galaxy in each the F450W and F814W filter. (For example the MGE we use for F814W is the MGE given in tab. \ref{tab:MGEF814W} convolved with the PSF in tab. \ref{tab:PSFMGEF814W}.) The lensing images become then visible in the residuals (see fig. \ref{fig:lens_just_imgpos}). Because the lensing images are extended, we use the position of the brightest pixel in each of the images. We also use the F814W-MGE subtracted residuals from \citet{SWELLSIII}. The lensing positions as determined from the latter are given in tab. \ref{tab:lenspos}. The scatter of lensing positions as determined from subtracting different surface brightness models from the galaxy in different filters gives an error of $\pm 1$ pixel on the image positions.

\begin{table}
\centering
\begin{minipage}{70mm}
\begin{tabular}{r|rrrr|c}
\hline
  & A & B & C & D & G\\\hline
$x_i$ [pixel] & 12.1 & -8.5 & 21.7 & -3.3 & 0.5 $\pm \sqrt{2}$ \\
$y_i$ [pixel] & 16.6 & -10.4 & -0.5 & 19.2 & 0.5 $\pm \sqrt{2}$ \\
\hline
\end{tabular}
\caption{Positions of the lensing images (A-D) and the galaxy center (G) in fig. \ref{fig:lens_just_imgpos}. The image positions were determined from the lens subtracted image for J1331 in figure 4 of \citet{SWELLSIII}, rotated to the $(x,y)$ coordinate system in fig. \ref{fig:lens_just_imgpos}. The pixel scale is 1 pixel = 0.05 arcsec and the error of each image position is $\pm$ 1 pixel. \textit{SMALL PROBLEM: Somehow I used $\sqrt(2)$ pixel as the error on the galaxy center in the Monte Carlo sampling code instead of the 0.5 pixel I claim here. Should I change this table and the error bars in the figures to match this bug????}}
\label{tab:lenspos}
\end{minipage}
\end{table}


\paragraph{Fitting.} As described in \S\ref{sec:lensmodel} our lensing model has the following free parameters: the source position $(\xi,\eta)$, and the radial slope $\alpha$ and Fourier parameters $(a_k,b_k)$ of the lens mass distribution in eq. (\ref{eq:scalefreemodel}) and (\ref{eq:Fourieransatz}). We want to find the lensing model which minimizes for all four images the distance between the observed image positions $\vec{\theta}_{oi}$ and those predicted by the lensing model $\vec{\theta}_{pi}$. Because we want to avoid solving the lens equation (cf. eq. (\ref{eq:lenseqpot}) and (\ref{eq:Fourierlenseq})) for $\theta_{pi}$, we follow \citet{1991ApJ...373..354K} and cast the calculation back to the source plane using the magnification tensor in eq. (\ref{eq:magnificationtensor}):
\begin{eqnarray*}
\chi^2_\text{lens} &=& \sum_{i=1}^{4} \left|\left( \begin{matrix} \frac{1}{\Delta_x} & 0\\0 & \frac{1}{\Delta_y}\end{matrix}\right) \left( \vec{\theta}_{pi} - \vec{\theta}_{oi} \right)\right|^2\\
&\simeq& \sum_{i=1}^{N} \left|\left( \begin{matrix} \frac{1}{\Delta_x} & 0\\0 & \frac{1}{\Delta_y}\end{matrix}\right)  \left.\mathscr{M}\right|_{\vec{\theta}=\vec{\theta}_{oi}} \left( \begin{matrix} \xi - \tilde{\xi}_i \\ \eta - \tilde{\eta}_i \end{matrix} \right) \right|^2,
\end{eqnarray*}
where $(\Delta_x,\Delta_y)$ are the measurement errors of the image positions $\vec{\theta}_{oi}$. $\left.\mathscr{M}\right|_{\vec{\theta}=\vec{\theta}_{oi}}$ is the magnification tensor and $(\tilde{\xi}_i,\tilde{\eta}_i)$ the source position according to the lens equation evaluated at $\vec{\theta}_{oi}$. Following \citet{GlennEC} we add a term
\begin{equation*}
\chi^2_\text{shape} = \lambda \sum_{k \geq 3} \frac{\left(a_k^2 +b_k^2 \right)}{a_0^2} 
\end{equation*}
which forces the shape of the mass distribution to be close to an ellipse. The total $\chi^2$ to minimize is therefore
\begin{equation*}
\chi^2 = \chi^2_\text{lens} + \chi^2_\text{shape}
\end{equation*}
We set $a_1 = b_1 = 0$, which corresponds to the choice of origin; in this case the center of the galaxy.
\\To be able to constrain the slope $\alpha$, we would have needed flux ratios for the images as in \citet{GlennEC}. But the extended quality of the images and the uncertainty in surface brightness subtraction makes flux determination too unreliable and we do not include them in the fitting. Even though the constraint from just the image position fit on $\alpha$ is very weak, we were however able to show that the image positions in tab. \ref{tab:lenspos} minimize $\chi^2$ at $\alpha=1$ and also our other image position sets from different models and filters are consistent with a flat rotation curve. In the following we therefore set $\alpha=1$.

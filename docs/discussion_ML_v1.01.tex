\subsection{Stellar mass-to-light ratios}



[TO DO] [TO DO: Also compare with results from Treu et al.]

\paragraph{Collection of thoughts}
\begin{itemize}
\item [TO DO: Start on page 62]
\end{itemize}

\begin{table}
\centering
\begin{tabular}{cccccc}
\hline\hline
& & \multicolumn{2}{c}{Chabrier IMF} & \multicolumn{2}{c}{Salpeter IMF}\\
      &  $L$ [$10^{10}L_{\odot}$]                & $M_*$ [$10^{10}M_\odot$]               & $\Upsilon_*^\text{chab}$ & $M_*$ [$10^{10}M_\odot$] & $\Upsilon_*^\text{sal}$ \\\hline
bulge &   $3.10 \pm 0.15 $  & $7.8 \pm 1.8$ & $2.5 \pm 0.6$ & $14.5 \pm 3.7 $ & $4.7 \pm 1.2$ \\
disk  &   $2.35 \pm 0.11 $  & $2.9 \pm 0.7$ & $1.2 \pm 0.3$ & $5.2 \pm 1.1$ & $2.2 \pm 0.5$ \\
total &   $5.45 \pm 0.19$ & $10.6 \pm 1.9$& & $19.7 \pm 3.9$&\\\hline
\end{tabular}
\caption{Total I-band luminosity, stellar mass and mass-to-light ratio, calculated from the I-band AB magnitudes and stellar masses found for J133's bulge and disk by \citet{SWELLS} (their table 2) for comparison with this work. The transformation from AB magnitudes to the Johnson-Cousins I-Band used the relation $I[\text{mag}] = I[\text{ABmag}] - 0.309$ from [TO DO: REF: Frei and Gunn 1994] (their table 2). For the conversion from apparent magnitude to total luminosity the redshift $z=0.113$ [TO DO: REF: Brewer] was turned into a luminosity distance using the cosmology by [TO DO: REF: Dunkley]. }
\label{tab:previousresults}
\end{table}
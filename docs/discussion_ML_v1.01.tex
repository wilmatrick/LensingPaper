We've presented different dynamical models of the central region of J1331, some of them capturing the observed kinematics well, but none of them work both at small and large radii. In the following we will discuss possible reasons for these deviations by also comparing our results to previous work.

\subsection{Stellar mass-to-light ratios}

Further knowledge of the stellar populations within J1331 and their initial mass functions (IMF) would be useful. In particular a sophisticated guess for the \emph{stellar} mass-to-light ratio in J1331's bulge could be compared to our very reliable measurement of the \emph{total} mass-to-light ratio inside the Einstein radius (see Table \ref{tab:bestfitlensmodel}) to support or contradict the presence of a lot of dark matter in the bulge. Traditional choices for the IMF are the bottom-heavy IMF by \citet{Salpeter1955},
$$\xi(m) \propto m^{-x}, x=2.35,$$
(where $\xi(m) \diff m$ is the number of stars with mass $m$ in $[m,m+\diff m]$) and the IMFs by \citet{2002Sci...295...82K} and \citet{Chabrier2003}, which are in agreement with each other and predict less low-mass stars. \\\citet{Ferreras} found a relation between the central stellar velocity dispersion $\sigma_0$ in early-type galaxies and the IMF slope. A higher $\sigma_0$ suggests a more bottom-heavy IMF. For an unimodal (Salpeter-like) IMF and a $\sigma_0 \propto 200 $ km/s for J1331 (see Figure \ref{fig:kinematics}) they predict $x \approx 2.33$ (see Figure 4 in \citet{Ferreras}), which is close to the standard Salpeter slope. These findings are supported by \citet{2014MNRAS.438.1483S}, who found for SDSS galaxies also a relation between unimodal (single power-law) IMF slope and stellar velocity dispersion. Their Table 2 lists $x(\sigma_*=190 \pm 20 \text{ km s}^{-1}) = 2.08 \pm 2.08$ and $x(\sigma_*= 230 \pm 20 \text{ km s}^{-1}) = 2.33 \pm 0.4$. When assuming a bi-modal (Kroupa-equivalent-like) IMF, \citet{Ferreras} predict $x \approx 2.85$ (see Figure 4 in \citet{Ferreras}) for J1331's central velocity dispersion. This is more bottom-heavy than the standard \citet{2002Sci...295...82K} IMF. Overall the central velocity dispersion suggests a rather bottom-heavy IMF J1331's bulge and therefore large stellar mass-to-light ratio.
\\ \citet{SWELLSI} estimated J1331's stellar bulge mass given a Salpeter IMF and measured the I-band AB magnitude of the bulge. Transformed to a stellar I-band mass-to-light ratio (see Table \ref{tab:previousresults}) this would correspond to $\Upsilon_\text{I,*}^\text{Salpeter} = 4.7 \pm 1.2$.
\\We compare our findings of the stellar mass-to-light ratios with the study by \cite{SWELLSV}: \cite{SWELLSV} found that the bulge of J1331 has an IMF more bottom-heavy than the Salpeter IMF from fitting a NFW halo to (1) the Einstein mass and (2) gas kinematics at larger radii $\gtrsim 8$ arcsec. We fitted the stellar kinematics within $\sim 3.5$ arcsec and the Einstein mass with a NFW halo and found $\Upsilon^\text{dyn}_{*,I} = 4.2 \pm 0.2$ (see Table \ref{tab:modelB4_bestfit}). \cite{SWELLSV}  found systematically lower NFW halos ($v_\text{circ}(5\text{arcsec}) \sim 120 \text{km s}^{-1}$ according to their Figure 2) than we did ($v_\text{circ}(5\text{arcsec}) \sim 200 \text{km s}^{-1}$, see Figure \ref{fig:kinematics}). [TO DO: Continue writing notes into text]
\begin{itemize}
\item We found an Einstein mass-to-light ratio of $\Upsilon_\text{I,tot,Ein} = 5.6$ (see Table \ref{tab:bestfitlensmodel}). In case of a lower mass NFW halo, the largest contribution to this mass-to-light ratio would then be stellar mass. And the corresponding IMF would be more bottom heavy than Salpeter as well, like what \citet{SWELLSV} found.
\item The light distribution multiplied with $\Upsilon_\text{I,tot,Ein}$ looks very similar to \cite{SWELLSV} [TO DO: Look details up in their paper and my work]
\item In the JAM model based on the surface brightness alone with the increasing mass-to-light ratio, we found a total mass-to-light ratio in the center of J1331 of  $\Upsilon_\text{I,tot}(R\sim0) = 2.53$. This is very close to the stellar mass-to-light ratio we got from \citet{SWELLSI} estimates of the bulge's stellar mass using the \citet{Chabrier2003} IMF $\Upsilon_\text{I,*}^\text{Chabrier} = 2.5 \pm 0.6$, i.e. less bottom-heavy (more blue) population.
\item The central dip of the stellar kinematics cannot be explained alone by a strong contribution of dark matter, because the best fit NFW halos are too massive to explain the kinematics at larger radii.
\item The central dip can also not be explained alone by tangential [TO DO: check???] velocity anisotropy in the center and a lower contribution by dark matter.
\item The above discussion motivates the following speculation: In the absence of a strong dark matter contribution in the center, the overall stellar-mass-to-light ratio within the Einstein radius indicates a bottom-heavy Salpeter-like IMF, consistent with estimates from the velocity dispersion. At the same time the central dip can then be only explained, if there was an increase in stellar mass-to-light ratio with radius \textit{within} the bulge. \emph{If} there was a stellar population in the very center with an IMF close to the \citet{Chabrier2003} IMF, surrounded by a more bottom-heavy population, the central stellar kinematics would be well explained and be fully consistent with the lensing results, as found in Figure \ref{fig:modelG} and the surrounding text. It is not expected that the bluish disk of J1331 has a $\Upsilon_{*,I}$ as high as the reddish bulge. The kinematics at larger radii would then be much better recovered by lower mass dark matter halo as found by \citet{SWELLSV}.
\item Standard JAM modelling approaches seem not to work for J1331. A JAM model for J1331 would need to allow for stellar mass-to-light ratio gradients within the galaxy, velocity anisotropy and a dark matter halo. Because of degeneracies between stellar and dark mass, and matter distribution and anisotropy profile, such a dynamical model would not lead to very tight constraints on the model parameters. Further priors are needed.
\item Detailed stellar population analyses of the spectra taken along J1331's major axis could possibly support or contradict the suspicion of the existence of stellar mass-to-light ratio gradients in J1331.
\end{itemize}

[TO DO: Rewrite the paper such, that it's main message is, that there is a strong indication of increasing stellar mass-to-light ratio in the center of the galaxy, i.e. a bluish core, because neither velocity anisotropy, neither a strong dark matter contribution alone can explain all dynamical signatures. "Evidence from Lensing and Dynamics for a bluish core in the center of J1331" as a possible alternative title??? Eher nicht...]

[TO DO] [TO DO: Also compare with results from Treu et al.]

\paragraph{Collection of thoughts}
\begin{itemize}
\item [TO DO: Start on page 62]
\end{itemize}

\begin{table}
\centering
\begin{tabular}{cccccc}
\hline\hline
& & \multicolumn{2}{c}{Chabrier IMF} & \multicolumn{2}{c}{Salpeter IMF}\\
      &  $L$ [$10^{10}L_{\odot}$]                & $M_*$ [$10^{10}M_\odot$]               & $\Upsilon_*^\text{chab}$ & $M_*$ [$10^{10}M_\odot$] & $\Upsilon_*^\text{sal}$ \\\hline
bulge &   $3.10 \pm 0.15 $  & $7.8 \pm 1.8$ & $2.5 \pm 0.6$ & $14.5 \pm 3.7 $ & $4.7 \pm 1.2$ \\
disk  &   $2.35 \pm 0.11 $  & $2.9 \pm 0.7$ & $1.2 \pm 0.3$ & $5.2 \pm 1.1$ & $2.2 \pm 0.5$ \\
total &   $5.45 \pm 0.19$ & $10.6 \pm 1.9$& & $19.7 \pm 3.9$&\\\hline
\end{tabular}
\caption{Total I-band luminosity, stellar mass and mass-to-light ratio, calculated from the I-band AB magnitudes and stellar masses found for J133's bulge and disk by \citet{SWELLS} (their table 2) for comparison with this work. The transformation from AB magnitudes to the Johnson-Cousins I-Band used the relation $I[\text{mag}] = I[\text{ABmag}] - 0.309$ from \citet{FG1994} (their table 2). For the conversion from apparent magnitude to total luminosity the redshift $z=0.113$ \citet{SWELLSIII} was turned into a luminosity distance using the cosmology by \ref{WMAP5cosm}. }
\label{tab:previousresults}
\end{table}
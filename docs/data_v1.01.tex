\section{Data} \label{sec:data}

[TO DO]

\begin{itemize}
\item Hubble Space teleskope (HST) imaging by \cite{SWELLSI} in two filters (F450W amd F814W)
\item I filter: for surface brightness distribution of J1331's bulge for Jeans modelling
\item ??? filter: to identify bluish lensing images
\item drizzled image
\end{itemize}

\paragraph{HST imaging.} [TO DO] Follow up high resolution imaging with the Hubble Space Telescope's (HST) Wide-Field Planetary Camera 2 (WFPC2) was performed \citep{SWELLSI}). The WF3 CCD chip pf WFPC2. The images we are using are a combination of four exposures with each an exposure of 400s and were drizzled to a pixel scale of 1 pixel = 0.05 arcsec. Filters F450W and F814W. Redshifts Lens image redshift $z_s = 0.254$  \citep{SWELLSIII} This galaxy is at a redshift of $z_d = 0.113$ (\cite{SWELLSIII}). Angular diameter distance according to [TO DO: REF: Dunkley] is 414 Mpc, which translates into a scaling of 1 arcsec $\hat{=}$ 2.01 kpc for J1331.


\paragraph{Stellar Kinematics.} We use the stellar kinematics for J1331 measured by [TO DO: REF: Dutton 2013]. They obtained long-slit spectra along J1331's major-axis with the Low Resolution Imaging Spectrograph (LRIS) on the Keck I 10m telescope. The width of the slot was 1 arcsec and the seeing Conditions had a FWHM of $\sim 1.1$ arcsec. Spectra for spatial bins of different widths along the major axis were extracted. Analogously to [TO DO: REF: Dutton 2011] they measured line-of-sight rotation velocities ($v_\text{rot}$) and stellar velocity dispersion ($\sigma$) by fitting Gaussian line profiles to emission lines in these spectra. Gas kinematics were extracted from fits to H$\alpha$ and NII lines, as tracers for inonized gas.
\\The stellar kinematics, $v_\text{rot}$, $\sigma$ and $v_\text{rms}^2=v_\text{rot}^2 + \sigma^2$ are shown in Figure \ref{fig:kinematics}. The rotation curve reveals a counter-rotating core within 2 arcsec $\simeq$ 4 kpc. Outside of $\sim 3.5 arcsec$ there is a steep drop in the dispersion, which is exptected at the boundary between the pressure supported bulge and the rotationally supported disk, which appears around this radius in the F450W filter in Figure \ref{fig:F450W}. However, in the brighter F814W filter in Figure \ref{fig:F814W}  the large reddish bulge extends out to $\sim$5 arcsec. 
\\Inside of $\sim 4$ arcsec, the data appears to be symmetric, outside of this the assumption of axisymmetry seems not to be valid anymore, considering the data. We add -2.3 km/s to the $v_\text{rot}$ to ensure $v_\text{rot}(R=0) \sim 0$ as a possible correction term for a systematic misjudgement of the systemic velocity. We also symmetrize the data within 4 arcsec and asign a minimum error of $\delta v_\text{rms} > 5$ km/s to the $v_\text{rms}$ data. In the JAM modelling, which is based on the assumption of axisymmetry, only kinematics within $\sim 2.5$ and 4 arcsec are used. Another reason to restrict to modelling on the bulge region is that our MGE in Table \ref{tab:MGEF814W} is only a good representation of J1331's F814W light distribution inside $\sim 5$ arcsec.
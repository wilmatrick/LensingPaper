\subsection{Summary}

%Intro:
J1331 is a massive spiral galaxy which acts as strong gravitational lens for a background source. J1331 appears also to be a merger remnant due to it's large counter-rotating core. The goal of this work was to recover J1331's matter distribution by means of two methods, gravitational lensing and dynamical modelling. We focus on the bulge region of J1331, to complement previous studies of J1331 by \citet{SWELLSIII,SWELLSV}. The independent mass constraint which we derive from lensing and dynamics agree very well with each other within $R_\text{eff}$.

In our lensing approach we fitted a scale-free galaxy model to the lensing image position. This constrained the Einstein radius to within 2\% ($R_\text{ein}=(0.91\pm0.02)''\simeq(1.83\pm0.04)~\text{kpc}$), and the Einstein mass to within 4\% ($M_\text{ein} = (7.8\pm0.3) \cdot 10^{10} M_\odot$), consistent with results by \citet{SWELLSIII}. The recovered mass distribution agrees also very well with the shape of the observed lensing images.

To get a handle of J1331's light distribution we used a MGE fit to J1331's surface brightness in the F814W filter in HST imaging by \citet{SWELLSI}. Based on this we found J1331's effective radius to be $R_\text{eff} \simeq 2.6'' \hat{=} 5.2~\text{kpc}$ and its total I-band luminosity to be $L_\text{I,tot} \simeq 5.6 \cdot 10^{10} L_\odot$.

We used axisymmetric JAM modelling to calculate the second velocity moment of a tracer and mass distribution given in terms of MGEs. The JAM model of the lens mass model gave a independent prediction which was consistent with observed stellar kinematics from major axis long slit spectrography by \citet{SWELLSV} within $R_\text{eff}$. We also fitted mass models with and without NFW halo to the stellar kinematics within $R'=3.5~\text{kpc}$. From this we deduced that a mass-follows-light model with constant total mass-to-light ratio is not a good model for J1331---not even in the inner regions. To describe the observed stellar kinematics we either require a strong contribution of a dark matter halo already in the bulge, or a strongly varying stellar mass-to-light ratio. We also showed that it is possible to construct a model which includes the counter-rotating core and is consistent with the rotation curve at large radii.

While both our independent mass models are consistent with each other within $\sim R_\text{eff}$, they do not agree with previous findings by \citet{SWELLSV}, nor the data at large radii. We attribute this to the fact \citet{SWELLSV} derived mass constraints from kinematics in the outer regions only, while we focussed on the inner regions in detail. We speculate that a merger might have modified the kinematic structure and/or mass distribution of J1331 such that crucial model assumptions are not valid anymore. To resolve the ambiguities in J1331's mass distribution two-dimensional kinematic maps of J1331 from integral-fiel units spectroscopy would be needed.
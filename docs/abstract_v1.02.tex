We analyse the stellar and dark matter distribution in the spiral galaxy SDSS J1331+3638 (J1331) by means of two independent methods: gravitational lensing and dynamical Jeans modelling. Hubble Space Telescope (HST) imaging by \citet{SWELLSI} reveals, that J1331's bulge is superimposed by a quadruplet of extended lensing images. By fitting a gravitational potential model to the image positions, we constrain the mass inside the Einstein radius ($R_\text{ein}=0.91\pm0.02$ arcsec) to within 4\% ($M_\text{ein} = (7.8\pm0.3) \cdot 10^{10} M_\odot$). From Multi-Gaussian Expansions (MGE) of J1331's surface brightness distribution we find that J1331 has a total luminosity of $L_{I,tot} \simeq 5.6 \cdot 10^{10}L_{I,\odot}$ and an effective radius of $R_\text{eff} \simeq 2.6$ arcsec = 5.6 kpc. [TO DO: apparent brightness is boring, right?]

According to the long-slit major axis stellar kinematics from \citet{SWELLSV}, J1331 has a counter-rotating stellar core inside $\sim 2$ arcsec. We model the observed stellar kinematics in J1331's central regions by finding MGE models for the stellar and dark matter distribution that solve the axisymmetric Jeans equations. We find that J1331 requires a steep total mass-to-light ratio gradient in the center to reproduce the observed stellar kinematics. The best fit dynamical model predicts a total mass inside the Einstein radius consistent with the lens model, and vice versa the lens model gives an successful prediction for the observed kinematics in the galaxy center. For a dynamical model including a NFW dark matter halo,  we constrain the halo to have virial velocity $v_{200} \simeq 240 \pm 40$ km/s and a concentration of $c_{200} \simeq 8 \pm 2$ in case of a moderate tangential velocity anisotropy of $\beta_z \simeq −0.4 \pm 0.1$. The NFW halo models can successfully reproduce the signatures of J1331's counter-rotating stellar core and predict J1331's rotation curve at larger radii. However, all these models were more massive than expected from the gas rotation curve at larger radii, and failed to reproduce the steep drop in measured velocity dispersion at [TO DO: WHAT RADIUS???]. This could indicate a non-trivial re-distribution of matter due a possible minor merger event in J1331's past.
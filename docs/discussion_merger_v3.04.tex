\subsection{On J1331's possible merger history and modelling failures}

%Why we think that J1331 had a merger
J1331 has a large counter-rotating stellar core within $\sim 2''$. This suggests a process in J1331's past in which two components with angular momenta oriented in opposite directions were involved.

%Gas accretion - no!
Accretion of gas on retrograde orbits and subsequent star formation could lead to a younger and counter-rotating stellar population \Wilma{[TO DO: REF]}. However, to form enough stars such that the net rotation of the large and massive core is retrograde, a very large amount of gas would have had to be accreted by J1331---which is not very likely. 

%Mergers - yes!
Another scenario are galaxy mergers. Major mergers can form kinematically decoupled cores (KDCs) (e.g., \citealt{2011MNRAS.414.2923K,2015ApJ...802L...3T}), if they include large amounts of gas \citep{2010ApJ...723..818H}. During a minor merger, the dense nucleus of a satellite galaxy on a retrograde orbit could survive the dissipationless accretion and spiral to the core due to tidal friction (e.g., \citealt{1984ApJ...287..577K}). 

%Merger + color gradient
Usually ellipticals and the bulges of massive spirals appear reddest in their center and getting increasingly bluer with larger radii. Mergers can reverse this behaviour by inducing the creation of young stellar populations in the remnant's center. Major mergers can trigger star formation bursts (e.g., \Wilma{[TO DO: REF, maybe http://arxiv.org/pdf/1409.5126v2.pdf???]}). After a minor merger the satellite's stellar population now residing in the remnant's core is in general younger than the progenitor's bulge \Wilma{[TO DO: REF]} . The different stellar populations in a merger remnant can be associated with different $\Upsilon_*$ and sometimes even show up as a reverse colour gradient within the bulge in photometry \Wilma{[TO DO: REF]}.

%Color gradient in J1331
Even though investigation of the photometry of J1331 did not reveal a distinct blue core in J1331, we cannot fully exclude the possibility that J1331 has such a $\Upsilon_*$ gradient (see discussion in Section  \Wilma{[TO DO: Where???]}). In this case the restrictive assumption of a constant $\Upsilon_*$ in our JAM modelling could have lead to wrong conclusions about the DM distribution.

%Merger and kinematic twist
Other ways how mergers can can modify the structure of galaxies are warps in the outer parts or kinematic twists in the center of the remnant galaxy \citep{2013pss5.book..923S}, which show up as misalignments of kinematic and photometric major axis.

%Kinematic twist in J1331
From kinematics along the photometric major axis only it cannot be determined if such a misalignement is present in J1331. But if it were, our modelling assumption of axisymmetry would not be satisfied anymore. 

We note that kinematics measured along an axis different from the kinematic major axis would result in $v_\text{rot}$ measurements being lower than the actual maximum $v_\text{circ}$. At large radii this could lead to an underestimation of the DM halo mass.
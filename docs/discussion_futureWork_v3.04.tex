\subsection{Future work}

Standard JAM modelling approaches seem not to work for J1331. The existing data---photometry and major axis kinematics---is however not sufficient to decide which of our model assumptions lead to the failure to find an overall consistent mass model.

Two-dimensional stellar kinematics would greatly help to resolve this problem. The main future work would be therefore getting high-resolution integral-field spectroscopy for J1331. High spatial resolution would be required to clearly identify J1331's presumably complex kinematic structure. High spectral resolution would be important to be able to reliably measure the low velocity dispersion in the outer regions of J1331.

Specifically 2D kinematics should help to answer the following questions: Is the drop in $v_\text{rms}$ around $R' \sim 3-6~\text{kpc}$ real? Did the long slit spectrograph maybe miss the major axis of the disk? And most importantly: Are the kinematics asymmetric? Is it possible that there even exists a kinematic twist due to the merger in J1331's past?

Furthermore, learning more about different stellar populations in J1331 would lead to valuable constraints for the modelling. While a quick investigation of the photometric colour profile did not reveal obvious colour gradients in J1331's bulge, there could be still stellar $\Upsilon_{I,*}$ variations due to age of metallicity differences. Existing major axis spectroscopy and/or future IFU data could be employed (i) to investigate absorption line indices to confirm (or contradict) the existence of $\Upsilon_{I,*}$ gradients and (ii) to perform stellar population analyses to constrain $\Upsilon_{I,*}$ reliably and independently of kinematics.

Future modelling approaches should fit dynamics (stellar and gas kinematics from) simultaneously with the gravitational lensing (image positions, shape and even flux ratios) in a similar fashion to \citet{SWELLSIV}. To also model the extent, shape and flux of the lensing images, the method by \citet{2004ApJ...611..739T,2003ApJ...590..673W} could be employed, which models the surface brightness distribution of the images and source on a pixelated grid. For this to work a good model for the galactic extinction would be needed---but 2D spectroscopy could also help with this.

All of the would lead to a much better understanding of J1331's structure and mass distribution and therefore answer questions on how mergers might modify spiral galaxies.
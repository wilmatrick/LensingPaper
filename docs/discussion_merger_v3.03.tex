\subsection{On J1331's possible merger history and resolution of modelling failures}

J1331 has a large counter-rotating stellar core within $\sim 2''$. This suggests a process in J1331's past in which two components with angular momenta oriented in opposite directions were involved. Accretion of gas on retrograde orbits and subsequent star formation could lead to a younger and counter-rotating stellar population \Wilma{[TO DO: REF]}. However, to form enough stars such that the net rotation of the large and massive core is retrograde, a very large amount of gas would have had to be accreted by J1331---which is not very likely [TO DO: REF]. 

Another scenario are galaxy mergers. Major mergers including large amounts of gas can form kinematically decoupled cores (KDCs), see e.g., \Wilma{[TO DO: REF: Lauren Hofman et al 2010 - read paper how this fits with the reverse color gradient]}. During a minor merger, the dense nucleus of a satellite galaxy on a retrograde orbit could survive the dissipationless accretion and spiral to the core due to tidal friction \citep{1984ApJ...287..577K}. 

Usually ellipticals and the bulges of massive spirals appear reddest in their center and getting increasingly bluer with larger radii. Mergers can reverse this behaviour in the remnant's core. A star formation burst triggered by the major merger could lead to a new young stellar population in the galaxy's bulge \Wilma{[TO DO: Check, if I can claim that]}. In case of a minor merger the satellite nucleus now residing in the merger remnant's very center is also much younger than the bulge stars of the massive progenitor, because star formation in low-mass satellite galaxies occurs over a longer period and later than in massive galaxies \Wilma{[TO DO: REF]}. The different stellar populations in a merger remnant can be associated with different $\Upsilon_*$ and sometimes even show up as a reverse colour gradient within the bulge in photometry \Wilma{[TO DO: REF]}.

Even though investigation of the photometry of J1331 did not reveal a clear and distinct blue core in J1331, we cannot fully exclude the possibility that J1331 has $\Upsilon_*$ gradient. And as discussed \Wilma{[TO DO: Where???]} an increasing $\Upsilon_*$ could explain the observed central $v_\text{rms}$ dip.

Another way how major mergers can modify the structure of galaxies is a kinematic twist \Wilma{[TO DO: REF]} or misalignment (warp) of kinematic and photometric major axis \Wilma{[TO DO: REF]}. In projection a kinematic twist shows up as a misalignment of the kinematic major axis of bulge and disk \Wilma{[TO DO: No idea if it works like this.]}. In both cases---kinematic twist and warp---th \Wilma{[TO DO: Continue]}

%---------------

If one or both of the merger effects mentioned above---$\Upsilon_*$ gradient in the bulge and misalignment of the kinematic major axes---were also the case for J1331, this would introduce complexities in our dynamical modelling that we cannot account for based alone on stellar kinematics along the photometric major axis.

We speculate that the modelling approaches in this work could indeed point towards such merger modifications in J1331. Investigation, but... 

%---------------

%satellite core has lower velocity dispersion.
% Kinematic twist due to merger. Bulge and disk have different kinematic major axis, and the measured velocities in the disk, which are off from the major axis, are much lower than the actual maximum vcirc. Could explain the dips in the outer regions.
%Another effect that minor mergers can have on galaxies are a misalignement (warp) of the photometric and kinematic major axis \Wilma{[TO DO: REF]}. Would this be the case for J1331 the assumption of axisymmetry is not true anymore and our dynamical modelling would fail.

Mergers also could have changed the 3D shape of the dark matter halo and the NFW halo was therefore not a good model for J1331's dark matter halo. We also tried to model J1331 with a cored logarithmic halo. However, due to degeneracies in the modelling, we were not able to either constrain the profile for a cored logarithmic halo, or to rule it out. 

\Wilma{[TO DO: Include in this discussion the findings from the color profile.]}

This could indicate that there might be a less bottom-heavy, more bluish population, with an IMF close to the Chabrier IMF, in the very center of J1331.

\Wilma{[TO DO: Remove minor in merger.]}

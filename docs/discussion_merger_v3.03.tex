\subsection{On J1331's possible merger history and modelling failures}

%Why we think that J1331 had a merger
J1331 has a large counter-rotating stellar core within $\sim 2''$. This suggests a process in J1331's past in which two components with angular momenta oriented in opposite directions were involved.

%Gas accretion - no!
Accretion of gas on retrograde orbits and subsequent star formation could lead to a younger and counter-rotating stellar population \Wilma{[TO DO: REF]}. However, to form enough stars such that the net rotation of the large and massive core is retrograde, a very large amount of gas would have had to be accreted by J1331---which is not very likely [TO DO: REF]. 

%Mergers - yes!
Another scenario are galaxy mergers. Major mergers including large amounts of gas can form kinematically decoupled cores (KDCs), see e.g., \Wilma{[TO DO: REF: Lauren Hofman et al 2010 - read paper how this fits with the reverse color gradient]}. During a minor merger, the dense nucleus of a satellite galaxy on a retrograde orbit could survive the dissipationless accretion and spiral to the core due to tidal friction \citep{1984ApJ...287..577K}. 

%Merger + color gradient
Usually ellipticals and the bulges of massive spirals appear reddest in their center and getting increasingly bluer with larger radii. Mergers can reverse this behaviour in the remnant's core. A star formation burst triggered by the major merger could lead to a new young stellar population in the galaxy's bulge \Wilma{[TO DO: Check, if I can claim that]}. In case of a minor merger the satellite nucleus now residing in the merger remnant's very center is also much younger than the bulge stars of the massive progenitor, because star formation in low-mass satellite galaxies occurs over a longer period and later than in massive galaxies \Wilma{[TO DO: REF]}. The different stellar populations in a merger remnant can be associated with different $\Upsilon_*$ and sometimes even show up as a reverse colour gradient within the bulge in photometry \Wilma{[TO DO: REF]}.

%Color gradient in J1331
Even though investigation of the photometry of J1331 did not reveal a distinct blue core in J1331, we cannot fully exclude the possibility that J1331 has $\Upsilon_*$ gradient. And as discussed \Wilma{[TO DO: Where???]} an increasing $\Upsilon_*$ could explain the observed central $v_\text{rms}$ dip. While $\Upsilon_*$ gradients could be easily included in JAM modelling, it would add much more complexity and degeneracies. We would need either more kinematic data and/or a sophisticated guess for the $\Upsilon_*$ gradient to constrain the model parameters reliably.

%Merger and kinematic twist
Another way how major mergers can modify the structure of galaxies is a kinematic twist \Wilma{[TO DO: REF]} or misalignment (warp) of kinematic and photometric major axis \Wilma{[TO DO: REF]}. A kinematic twist shows up as a misalignment of the kinematic major axis of bulge and disk in projection \Wilma{[TO DO: No idea if it works like this.]}. 

%Kinematic twist in J1331
In both cases---kinematic twist and warp in J1331---the assumption of axisymmetry in our dynamical modelling would not be satisfied anymore. If such a kinematic twist is present in J1331 cannot be identified from kinematics along the photometric major axis alone, but should be immediately obvious in 2D kinematic maps. We note that if the kinematic major axis was off from the photometric major axis in the outer regions of J1331, the measured $v_\text{rot}$ in the disk \Wilma{[TO DO: some sinus i]} would be much lower than the actual maximum $v_\text{circ}$. This could maybe also explain the so far inexplicable dips in $v_\text{rms}$ at the transition from bulge to disk. It is therefore not completely ruled out, that J1331 might have a more massive DM halo than \citet{SWELLSV} found.

%Merger and halo in J1331
Mergers also could have changed the 3D shape of the DM halo and the NFW halo would be therefore not a good model for J1331's DM halo. We also tried to model J1331 with a cored logarithmic halo. However, due to degeneracies in the modelling, we were not able to either constrain the profile for a cored logarithmic halo, or to rule it out. 


%---------------

\Wilma{[TO DO: include somewhere "satellite core has lower velocity dispersion."]}

\subsection{Surface brightness model} \label{sec:MGE_theo}

\paragraph{Multi-Gaussian Expansion (MGE).} MGEs are used to parametrize the observed surface brightness or projected total mass of a galaxy as a sum of $N$ two-dimensional, elliptical Gaussians \citep{1991ApJ...366..599B,1992A&A...253..366M,1994A&A...285..723E,1999MNRAS.303..495E}. This work makes use of the algorithm and code\footnote{The IDL code package for fitting MGEs to images by \citet{Cap02} is available online at \url{http://www-astro.physics.ox.ac.uk/~mxc/software}. The version from June 2012 was used in this work.} by \citet{Cap02}. We assume all Gaussians to have the same center and position angle $\phi$, i.e. orientation of the Gaussians' major axis measured counter-clockwise from the $y'$-axis of the coordinate system with polar coordinates $(R',\theta')$. Then the surface brightness can be written as
\begin{eqnarray}
I(R',\theta') &=& \sum_{i=1}^{N} I_{0,i} \exp\left[ - \frac{1}{2\sigma_i^2} \left(x^{'2} + \frac{y^{'2}}{q_i^{'2}}\right)\right]\label{eq:MGEgeneral}\\
\text{with } I_{0,i} &=& \frac{L_i}{2\pi \sigma_i^2 q'_i}\label{eq:centralItotalL}\\
\text{and } x'_i &=& R' \cos(\theta' - \phi)\nonumber\\
y'_i &=& R' \sin(\theta' - \phi),\nonumber
\end{eqnarray}
where $I_{0,i}$ is the central surface brightness of each Gaussian, $L_i$ its total luminosity, $\sigma_i$ its dispersion along the major axis and $q'_i$ the axis ratio between the elliptical Gaussians major and minor axis.

We can also expand the telescopes point-spread function (PSF) as a sum of circular Gaussians,
\begin{equation}
\text{PSF}(x',y') = \sum_j \frac{G_j}{2 \pi \delta_j^2} \exp\left[- \frac{1}{2 \delta_j^2} \left({x'}^2 + {y'}^2 \right)\right], \label{eq:PSFgeneral}
\end{equation}
where $\sum_j G_j = 1$ and $\delta_j$ are in this case the dispersions of the circular PSF Gaussians. In this case the observed surface brightness distribution is a convolution of the intrinsic surface brightness in Equation \ref{eq:MGEgeneral} with the PSF in Equation \ref{eq:PSFgeneral}: $(I \ast \text{PSF}) (x',y')$ is then again a sum of Gaussians and can be directly fitted to an image of the galaxy in question.

$I(R',\theta')$ describes the intrinsic, to 2D projected light distribution or surface density of the galaxy. Under the assumption that the galaxy is oblate and axisymmetric, and given the inclination angle $i$ of the galaxy with respect to the observer, MGEs allow an analytic deprojection of the 2D MGE to get a 3D light distribution or density $\nu(R,z)$ for the galaxy,
\begin{equation}
\nu(R,z) = \sum_i \nu_{0,i} \exp \left[-\frac{1}{2\sigma_i}\left(R^2 + \frac{z^2}{q_i^2} \right) \right]. \label{eq:deprojMGE}
\end{equation}
The flattening of each axisymmetric 3D Gaussian $q_i$ and its central density $\nu_{0,i}$ follow from the observed 2D axis ratio $q'_i$ and surface density $I_{0,i}$ as
\begin{eqnarray*}
q_i^2 &=& \frac{q_i'^2 - \cos^2 i}{\sin^2 i}\\
\nu_{0,i} &=& \frac{q_i' I_{0,i}}{q_i \sqrt{2 \pi \sigma_i^2}}.
\end{eqnarray*}



\subsection{Jeans Axisymmetric Modelling (JAM)} \label{sec:model_JAM}

\paragraph{JAM modelling.} Jeans axisymmetric models (JAM) assume galaxies (a) to be collisionless, i.e. the collisionless Boltzmann equation for the distribution function $f(\vect{x},\vect{v},t)$ has to be satisfied ($\frac{\diff f(\vect{x},\vect{v},t}{\diff t} = 0$), (b) in a steady state ($\frac{\partial}{\partial t} = 0$), (c) axisymmetric (best described in cylindrical coordinates $(R,z,\phi)$ and $\frac{\partial}{\partial \phi} = 0$). From this follow the axisymmetric Jeans equations as the vector-valued first moment of the Boltzmann equation, i.e.
\begin{equation*}
\int \frac{\diff f}{\diff t} \Diff3 v = 0.
\end{equation*}
To be able to solve the Jeans equations, additional assumptions about the velocity ellipsoid tensor $\langle v_i v_j\rangle$ have to be made. We follow \citet{Cap08} and assume firstly, that the galaxy's velocity ellipsoid is aligned with the cylindrical coordinate system, i.e. $\langle v_i v_j\rangle = 0$ for $i\neq j$. Secondly, we assume a constant ratio between the radial and vertical 2nd velocity moments, 
\begin{equation}
\beta_z \equiv 1 - \langle v_z^2 \rangle / \langle v_R^2\rangle. \label{eq:bz}
\end{equation}
This reduces the Jeans equations to two equations for $\langle v_z^2 \rangle$ and $\langle v_\phi^2 \rangle$, that can be solved by means of one integration,
\begin{eqnarray}
n \langle v_z^2 \rangle (R,z) &=& \int_0^\infty n \frac{\partial \Phi}{\partial z} \diff z \label{eq:Jeans1}\\
n \langle v_\phi^2 \rangle (R,z) &=& R \frac{\partial}{\partial R} \left( \frac{n \langle v_z^2 \rangle}{1-\beta_z} \right) + \frac{n \langle v_z^2 \rangle}{1 - \beta_z} + R n \frac{\partial \Phi}{\partial R},\label{eq:Jeans2}
\end{eqnarray}
where $n(\vect{x}) = \int f(\vect{x},\vect{v}) \Diff3 v$ is the number density of tracers and $\Phi(\vect{x})$ the galaxy's gravitational potential, generated by the mass density $\rho(\vect{x})$ via Poisson's equation.
\\The JAM modelling approach by \citet{Cap08} makes use of expressing the tracer density and the mass density as MGEs (see also \citet{1994A&A...285..723E}). The density of stellar tracers is assumed to be proportional to the observed and deprojected brightness distribution $\nu(R,z)$ in Equation \ref{eq:deprojMGE}. The mass density $\rho(R,z)$ can consist of several sets of MGEs, describing stellar and dark matter components. The gravitational potential is generated from the mass MGE by integrating the Poisson equation \citep{1994A&A...285..723E}. Equations \ref{eq:Jeans1} and \ref{eq:Jeans2} together with \ref{eq:bz} provide the velocity dispersion tensor $\langle v_i v_j \rangle = \langle v_i^2 \rangle$ (with $i,j \in \{ R,\phi, z\}$). It is then rotated by the inclination angle $i$ to the coordinate system of the observer ($(x',y')$ being the plane of the sky and $z'$ the line-of-sight, where $x'$ is alinged with the galaxy's major axis). Taking a light-weighted projection along the line-of-sight gives a model prediction for the line-of-sight velocity second moment, which is comparable to actual spectroscopic measurements of the second velocity moment. Details of the derivation using the MGE formalism are given in \citet{Cap08} and repeated in \citet{GlennEC}, we therefore just quote the result by \citet{Cap08} (their Equation 28) for the line-of-sight second velocity moment prediction from the Jeans equations,
\begin{eqnarray}
&&\left(I \langle v_\text{los}^2\rangle\right)(x',y')\nonumber\\
&&= 4 \pi^{3/2} G \int_0^1 \sum_{k=1}^N \sum_{j=1}^M \nu_{0,k} \ q_j \ \rho_{0,j} \ u^2\nonumber\\
&& \times \frac{\sigma_k^2 \ q^{2}_k \left( \cos^2 i + \frac{\sin^2 i}{1-\beta_{z,k}}\right) + \mathscr{D} \  {x'}^{2} \sin^2 i}{(1-\mathscr{C}u^2) \sqrt{(\mathscr{A} + \mathscr{B} \cos^2 i) \left[1-(1-q^{2}_j) u^2 \right]}}\nonumber\\
&& \times \exp\left\{- \mathscr{A} \left[{x'}^{2} + \frac{(\mathscr{A}+\mathscr{B}) \ {y'}^{2}}{\mathscr{A}+\mathscr{B}\cos^2 i}\right] \right\} \diff u, \label{eq:explicitLOSvelCap}
\end{eqnarray}
with $N$ Gaussians describing the tracer distribution and $M$ Gaussians describing the mass distribution, $\rho_{0,j}$ being the $j$-th mass density Gaussian evaluated at $(R=0,z=0)$ and
\begin{eqnarray*}
\mathscr{A} &=& \frac 12 \left(\frac{u^2}{\sigma_j^2} + \frac{1}{\sigma_k^2} \right)\nonumber\\
\mathscr{B} &=& \frac 12 \left\{\frac{1-q^{2}_k}{\sigma_k^2 q^{2}_k} + \frac{(1-q^{2}_j)u^4}{\sigma_j^2 \left[1-(1-{q}_j^{2})u^2 \right]} \right\}\nonumber\\
\mathscr{C} &=& 1- q^{2}_j - \frac{\sigma_k^2 q^{2}_k}{\sigma^2_j}\nonumber\\
\mathscr{D} &=& 1 - \frac{q^{2}_k}{1-\beta_{z,k}} - \left[ \left(1-\frac{1}{1-\beta_{k,z}}\right)\mathscr{C} + \frac{1-q^{2}_j}{1-\beta_{z,k}}\right] u^2.\nonumber
\end{eqnarray*}
The JAM modelling code by \citet{Cap08} evaluates numerically Equation \ref{eq:explicitLOSvelCap} for a given luminous tracer and mass distribution MGE and a given inclination.


\paragraph{Data comparison.} As data we use stellar line-of-sight rotation velocities $v_\text{rot} \equiv \langle v_\text{los} \rangle$ and velocity dispersions $\sigma$ as described in Section \ref{sec:data}. The JAM models give us a prediction for the second line-of-sight velocity moment $\langle v_\text{los}^2 \rangle$. The root mean square (rms) line-of-sight velocity $v_\text{rms}$ allows a data-model comparison by relating theses velocities according to 
\begin{equation*}
v_\text{rms}^2 = \langle v_\text{los}^2 \rangle = v_\text{rot}^2 + \sigma^2.
\end{equation*}
The model in Equation \ref{eq:explicitLOSvelCap} predicts the intrinsic $\langle v_\text{los}^2 \rangle$ at a given position on the sky, which have then to be modified according to the mode of observation, to be comparable to the measurements. The measured $v_\text{rms}$ is a light-weighted mean for a pixel along the long-slit of the spectrograph, with height $L_y = 1$ arcsec \citep{SWELLSV} and a certain given extent in along the galaxy's major axis, $L_x$, i.e. for a rectangular aperture
\begin{equation*}
\text{AP}(x',y') = \left\{ \begin{array}{ll} 1 & \text{for } -\frac{L_x}{2} \leq x' \leq + \frac{L_x}{2}\\
& \text{and } - \frac{L_y}{2} \leq y' \leq + \frac{L_y}{2}  \\ 0 & \text{otherwise.} \end{array} \right.
\end{equation*}
The light arriving at the spectrograph itself was subject to seeing, i.e. a Gaussian with Full Width Half Maximum (FWHM) of 1.1 arcsec \citep{SWELLSV}
\begin{equation*}
\text{PSF}(x',y')=\mathscr{N}(0,\text{FWHM}/2\sqrt{2\ln2}).
\end{equation*}
The model predictions have therefore to be convolved with the convolution kernel
\begin{eqnarray*}
K(x',y') &=& (\text{PSF} \ast \text{AP})(x',y') \\
&=& \frac{1}{4} \prod_{u \in \{x',y'\}} \left[ \text{erf}\left( \frac{L_u/2 - u}{\sqrt{2}\sigma_\text{seeing}}\right) + \text{erf} \left( \frac{L_u/2 + u}{\sqrt{2} \sigma_\text{seeing}} \right) \right]
\end{eqnarray*}
and weighted by the surface brightness $I(x',y')$
\begin{eqnarray*}
I_\text{obs} &=& I \ast K\\
\langle v_\text{los}^2 \rangle |_\text{obs} &=& \frac{(I \langle v_\text{los}^2\rangle) \ast K}{I_\text{obs}}.
\end{eqnarray*}
If provided with the convolution kernel, the JAM code\footnote{The IDL code package for Jeans Axisymmetric Modelling (JAM) by \citet{Cap08} is available online at \url{http://www-astro.physics.ox.ac.uk/~mxc/software}. The version from June 2012 was used in this work.} by \citet{Cap08} performs the convolution numerically. We set $L_x = 0.21$ arcsec as the width of the model pixel, and get a prediction for the actual measurements in bins of width 0.63, 1.26 and 1.89 arcsec \citep{SWELLSV} as light-weighted mean from each 3, 6 and 9 model pixels.

\paragraph{Rotation curve.} The intrinsic rotation curve is the first velocity moment $\langle v_\phi\rangle = \sqrt{\langle v_\phi^2 \rangle - \sigma_\phi^2}$. The observed rotation curve is the projection of the light-weighted contributions to $\langle v_\phi\rangle$ along the line-of-sight(\cite{Cap08}),
\begin{equation*}
I \langle v_\text{los}\rangle = \int_{-\infty}^{+\infty} \nu \langle v_\phi\rangle \cos \phi \sin i \diff z'.
\end{equation*}
The first velocity moments cannot be uniquely determined from the Jeans equations, which give only a prediction for the second velocity moments. Further assumptions are needed to separate the second velocity moments into ordered and random motion. \citet{Cap08} assumes that in a steady state there is no streaming velocity in $R$ direction, i.e. $\langle v_R \rangle = 0$ and therefore $\sigma_R^2 = \langle v_R^2 \rangle$. Then \citet{Cap08} relates the dispersions in $R$ and $\phi$ direction such that
$$\langle v_\phi\rangle = \sqrt{\langle v_\phi^2 \rangle - \sigma_\phi^2} \equiv \kappa \sqrt{\langle v_\phi^2 \rangle - \langle v_R^2 \rangle},$$
and the $\kappa$ parameter quantifies the rotation: $\kappa = 0$ means no rotation at all and $|\kappa| = 1$ describes a velocity dispersion ellipsoid that is a circle in the $R$-$\phi$ plane \citep{Cap08}. The sign of $\kappa$ determines the rotation direction. We can assign a constant $\kappa_k$ to every Gaussian in the MGE formalism and
\begin{equation*}
\nu \langle v_\phi\rangle = \left[\nu \sum_{k} \kappa_k^2 \left( [\nu\langle v_\phi^2 \rangle]_k - [\nu\langle v_R^2 \rangle]_k\right) \right]^{1/2}
\end{equation*} 
is then the light-weighted circular velocity curve, given the second velocity moments found from the Jeans equations.
\\To model the counter-rotating core of J1331 with one free parameter for, we employ the condition that the overall $\kappa(R)$ profile should smoothly and relatively steeply transition from $\kappa(R) = -\kappa' < 0$ at small $R$ through $\kappa(R_0) \equiv 0$ and increase to $\kappa(R) = \kappa' > 0$ at large $R$. Our imposed profile is
\begin{equation}
\kappa(R) = \kappa' \frac{R^2 - R_0^2}{R^2 + R_0^2}. \label{eq:kappa_profile}
\end{equation}
We find $\kappa'$ by matching the model $\langle v_\text{los} \rangle$ with the symmetrized $v_\text{rot}$ data, where for a given $\kappa'$ the $\kappa_k$ are found from fitting the MGE generated profile $\kappa(R) = \sum_k \kappa_k \nu_k(r)/\sum_k \nu_k(r)$ to Equation \ref{eq:kappa_profile}. The observed zero point is at $R'_0\approx 2$ arcsec. In the deprojected galactic plane the radius of zero rotation would be at a $R_0 \gtrsim 2$ arcsec, and we choose it to be at 2.2 arcsec.

\paragraph{Including a NFW halo.} As mentioned above, JAM modelling allows to incorporate an invisible matter component in addition to the luminous matter in the form of an MGE. In Section \ref{sec:results_JAM_NFW} we will include a spherical Navarro-Frenck-White (NFW) dark matter halo \citep{Navarro+1995c,NFW96} in the dynamical model. The classical NFW profile has the form
\begin{equation}
\rho_\text{NFW}(r) \propto \frac{1}{\frac{r}{r_s} \left( 1 + \frac{r}{r_s} \right)^2} \label{eq:NFWprofile}
\end{equation}
and two free parameters, the scale length $r_s$ and a parameter describing the total mass of the halo. We will use $v_\text{200}$, which is the circular velocity at the radius $r_\text{200}$ within which the mean density of the halo is 200 times the cosmological critical density $\rho_\text{crit} \equiv (2H^2)/(8\pi G)$, i.e.
\begin{eqnarray*}
M_\text{200} &=& M(<r_{200})\\
\frac{M_{200}}{ \frac 43 \pi r_{200}^3} &=& 200 \rho_\text{crit}(z=0) \\
v_\text{200} &=& \sqrt{\frac{GM_{200}}{r_\text{200}}}
\end{eqnarray*}
with $\rho_\text{crit}(z=0)=1.43 \cdot 10^{-7} M_\odot / \text{pc}^3$ in the WMAP5 cosmology by \citet{WMAP5cosm}. How much the mass is concentrated in the center of the NFW halo is given by the concentration of the NFW halo defined by 
\begin{equation}
c_{200}\equiv r_{200} / r_s. \label{eq:NFW_c}
\end{equation} 
There is a close relation between the concentration and halo mass in simulations \citep{NFW96}. \citet{Maccio08} found this relation for the WMAP5 cosmology \citep{WMAP5cosm} to be
\begin{equation}
\langle \log c_{200} \rangle (M_{200}) = 0.830 - 0.098 \log \left(h \frac{M_{200}}{10^{12} M_\odot} \right) \label{eq:Maccio08}
\end{equation}
(their equation 10), with a Gaussian scatter of $\sigma_{\log c_{200}} = 0.105$ (their Table A2). 
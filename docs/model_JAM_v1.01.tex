\subsection{Jeans Axisymmetric Modelling (JAM)} \label{sec:model_JAM}

Jeans axisymmetric models (JAM) assume galaxies (a) to be collisionless, i.e. the collisionless Boltzmann equation for the distribution function $f(\vect{x},\vect{v},t)$ has to be satisfied ($\frac{\diff f(\vect{x},\vect{v},t}{\diff t} = 0$), (b) in a steady state ($\frac{\partial}{\partial t} = 0$), (c) axisymmetric (best described in cylindrical coordinates $(R,z,\phi)$ and $\frac{\partial}{\partial \phi} = 0$). From this follow the axisymmetric Jeans equations as the vector-valued first moment of the Boltzmann equation, i.e.
\begin{equation*}
\int \frac{\diff f}{\diff t} \Diff3 v = 0.
\end{equation*}
To be able to solve the Jeans equations, additional assumptions about the velocity ellipsoid tensor $\langle v_i v_j\rangle$ have to be made. We follow \citet{Cap08} and assume firstly, that the galaxy's velocity ellipsoid is aligned with the cylindrical coordinate system, i.e. $\langle v_i v_j\rangle = 0$ for $i\neq j$. Secondly, we assume a constant ratio between the radial and vertical 2nd velocity moments, $\beta_z \equiv 1 - \langle v_z^2 \rangle / \langle v_R^2\rangle$. This reduces the Jeans equations to two equations for $\langle v_z^2 \rangle$ and $\langle v_\phi^2 \rangle$, that can be solved by means of one integration,
\begin{eqnarray*}
n \langle v_z^2 \rangle (R,z) &=& \int_0^\infty n \frac{\partial \Phi}{\partial z} \diff z\\
n \langle v_\phi^2 \rangle (R,z) &=& R \frac{\partial}{\partial R} \left( \frac{n \langle v_z^2 \rangle}{1-\beta_z} \right) + \frac{n \langle v_z^2 \rangle}{1 - \beta_z} + R n \frac{\partial \Phi}{\partial R},
\end{eqnarray*}
where $n(\vect{x}) = \int f(\vect{x},\vect{v}) \diff3 v$ is the number density of tracers and $\Phi(\vect{x})$ the galaxy's gravitational potential, generated by the mass density $\rho(\vect{x})$ via Poisson's equation.
\\The JAM modelling approach by \citet{Cap08} makes use of expressing the tracer density and the mass density as MGEs (see also \citet{1994A&A...285..723E}). The density of stellar tracers is assumed to be proportional to the observed and deprojected brightness distribution $\nu(R,z)$ in Eq. (\ref{eq:deprojMGE}). The mass density consists of several sets of MGEs: One MGE, that is usually taken to be $\nu(R,z)$ multiplied by a constant stellar mass-to-light ratio $\Upsilon_*$, describes the distribution of stellar mass in the galaxy. To mimick gradients of stellar mass-to-light ratio, each Gaussian could be assigned its own $\Upsilon_{*,i}$. To add a Navarro-Frenck-White (NFW) \citep{Navarro+1995c,NFW96} dark matter halo component, a MGE generated from a fit to a NFW profile can be added to the stellar component. [TO DO: continue on p. 74] 


\paragraph{Data comparison.} As data we use stellar line-of-sight rotation velocities $v_\text{rot} \equiv \langle v_\text{los} \rangle$ [TO DO: consistent, los or rot] and velocity dispersions $\sigma$ as described in \S\ref{sec:data}. The JAM models give us a prediction for the second line-of-sight velocity moment $v_\text{los}$. The root mean square (rms) line-of-sight velocity $v_\text{rms}$ allows a data-model comparison by relating theses velocities according to 
\begin{equation*}
v_\text{rms}^2 = \langle v_\text{los}^2 \rangle = v_\text{rot}^2 + \sigma^2.
\end{equation*}
The model in Eq. [TO DO] predicts the intrinsic $\langle v_\text{los}^2 \rangle$ at a given position on the sky, which have then to be modified to model the mode of observation, to be comparable to the measurements. The measured $v_\text{rms}$ is a light-weighted mean for a pixel along the long-slit of the spectrograph, with height $L_y = 1$ arcsec \citep{SWELLSV} and a certain given extent in along the major axis, $L_x$, i.e. for a rectangular aperture
\begin{equation*}
\text{AP}(x,y) = \left\{ \begin{array}{ll} 1 & \text{for } -\frac{L_x}{2} \leq x \leq + \frac{L_x}{2} \text{ and } - \frac{L_y}{2} \leq y \leq + \frac{L_y}{2}  \\ 0 & \text{otherwise} \end{array} \right. .
\end{equation*}
The light arriving at the spectrograph itself was subject to seeing, i.e. a Gaussian 
\begin{equation*}
\text{PSF}(x,y)=\mathscr{N}(0,FWHM/2\sqrt{2\ln2})
\end{equation*}
with FWHM=1.1 arcsec  \citep{SWELLSV}. The model predictions have therefore to be convolved with the convolution kernel
\begin{eqnarray*}
K(x,y) &=& (\text{PSF} \ast \text{AP})(x,y) \\
&=& \frac{1}{4} \prod_{u \in \{x,y\}} \left[ \text{erf}\left( \frac{L_u/2 - u}{\sqrt{2}\sigma_\text{seeing}}\right) + \text{erf} \left( \frac{L_u/2 + u}{\sqrt{2} \sigma_\text{seeing}} \right) \right]
\end{eqnarray*}
and weighted by the surface brightness $I(x,y)$ [TO DO: primed x and y or not????]
\begin{eqnarray*}
I_\text{obs} &=& I \ast K\\
\langle v_\text{los}^2 \rangle |_\text{obs} &=& \frac{(I \langle v_\text{los}^2\rangle) \ast K}{I_\text{obs}}.
\end{eqnarray*}
If provided with with the convolution kernel, the JAM code by \citet{Cap08} [TO DO: reference code] performs the convolution numerically. We set $L_x = 0.21$ arcsec as the width of the model pixel, and get a prediction for the actual measurements in bins of width 0.63, 1.26 and 1.89 arcsec \citep{SWELLSV} as light-weighted mean from each 3, 6 and 9 model pixels.

\paragraph{Rotation curve.} The intrinsic rotation curve is the first velocity moment $\langle v_\phi\rangle = \sqrt{\langle v_\phi^2 \rangle - \sigma_\phi^2}$. The observed rotation curve is the projection of the light-weighted contributions to $\langle v_\phi\rangle$ along the line-of-sight(\cite{Cap08}),
\begin{equation*}
I \langle v_\text{los}\rangle = \int_{-\infty}^{+\infty} \nu \langle v_\phi\rangle \cos \phi \sin i \diff z'.
\end{equation*}
The first velocity moments cannot be uniquely determined from the Jeans equations, which give only a prediction for the second velocity moments. Further assumptions are needed to separate the second velocity moments into ordered and random motion. \citet{Cap08} assumes that in a steady state there is no streaming velocity in $R$ direction, i.e. $\langle v_R \rangle = 0$ and therefore $\sigma_R^2 = \langle v_R^2 \rangle$. Then \citet{Cap08} relates the dispersions in $R$ and $\phi$ direction such that
$$\langle v_\phi\rangle = \sqrt{\langle v_\phi^2 \rangle - \sigma_\phi^2} \equiv \kappa \sqrt{\langle v_\phi^2 \rangle - \langle v_R^2 \rangle},$$
and the $\kappa$ parameter quantifies the rotation: $\kappa = 0$ means no rotation at all and $|\kappa| = 1$ describes a velocity dispersion ellipsoid that is a circle in the $R$-$\phi$ plane \citep{Cap08}. The sign of $\kappa$ determines the rotation direction. We can assign a constant $\kappa_k$ to every Gaussian in the MGE formalism and
\begin{equation*}
\nu \langle v_\phi\rangle = \left[\nu \sum_{k} \kappa_k^2 \left( [\nu\langle v_\phi^2 \rangle]_k - [\nu\langle v_R^2 \rangle]_k\right) \right]^{1/2}
\end{equation*} 
is then the light-weighted circular velocity curve, given the second velocity moments found from the Jeans equations.
\\To model the counter-rotating core of J1331 with one free parameter for, we employ the condition that the overall $\kappa(R)$ profile should smoothly and relatively steeply transition from $\kappa(R) = -\kappa' < 0$ at small $R$ [TO DO: Check, ich glaube das hier ist das intrinsische R in Zylinder-Coordinaten] trough $\kappa(R_0) \equiv 0$ and increase to $\kappa(R) = \kappa' > 0$ at large $R$. Our imposed profile is
\begin{equation}
\kappa(R) = \kappa' \frac{R^2 - R_0^2}{R^2 + R_0^2}. \label{eq:kappa_profile}
\end{equation}
We find $\kappa'$ by matching the model $\langle v_\text{los} \rangle$ with the symmetrized $v_\text{rot}$ data, where for a given $\kappa'$ the $\kappa_k$ are found from fitting the MGE generated profile $\kappa(R) = \sum_k \kappa_k \nu_k(r)/\sum_k \nu_k(r)$ to Eq. (\ref{eq:kappa_profile}). The observed zero point is at $R'_0\approx 2$ arcsec. In the deprojected galactic plane the radius of zero rotation would be at a $R_0 \gtrsim 2$ arcsec, and we choose it to be at 2.2 arcsec.
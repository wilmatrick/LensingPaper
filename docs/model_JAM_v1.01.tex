\subsection{Jeans Axisymmetric Modelling (JAM)}

Jeans axisymmetric models (JAM) assume galaxies (a) to be collisionless, i.e. the collisionless Boltzmann equation for the distribution function $f(\vect{x},\vect{v},t)$ has to be satisfied ($\frac{\diff f(\vect{x},\vect{v},t}{\diff t} = 0$), (b) in a steady state ($\frac{\partial}{\partial t} = 0$), (c) axisymmetric (best described in cylindrical coordinates $(R,z,\phi)$ and $\frac{\partial}{\partial \phi} = 0$). From this follow the axisymmetric Jeans equations as the vector-valued first moment of the Boltzmann equation, i.e.
\begin{equation*}
\int \frac{\diff f}{\diff t} \Diff3 v = 0.
\end{equation*}
To be able to solve the Jeans equations, additional assumptions about the velocity ellipsoid tensor $\langle v_i v_j\rangle$ have to be made. We follow \citet{Cap08} and assume firstly, that the galaxy's velocity ellipsoid is aligned with the cylindrical coordinate system, i.e. $\langle v_i v_j\rangle = 0$ for $i\neq j$. Secondly, we assume a constant ratio between the radial and vertical 2nd velocity moments, $\beta_z \equiv 1 - \langle v_z^2 \rangle / \langle v_R^2\rangle$. This reduces the Jeans equations to two equations for $\langle v_z^2 \rangle$ and $\langle v_\phi^2 \rangle$, that can be solved by means of one integration,
\begin{eqnarray*}
n \langle v_z^2 \rangle (R,z) &=& \int_0^\infty n \frac{\partial \Phi}{\partial z} \diff z\\
n \langle v_\phi^2 \rangle (R,z) &=& R \frac{\partial}{\partial R} \left( \frac{n \langle v_z^2 \rangle}{1-\beta_z} \right) + \frac{n \langle v_z^2 \rangle}{1 - \beta_z} + R n \frac{\partial \Phi}{\partial R},
\end{eqnarray*}
where $n(\vect{x}) = \int f(\vect{x},\vect{v}) \diff3 v$ is the number density of tracers and $\Phi(\vect{x})$ the galaxy's gravitational potential, generated by the mass density $\rho(\vect{x})$ via Poisson's equation.
\\The JAM modelling approach by \citet{Cap08} makes use of expressing the tracer density and the mass density as MGEs (see also [TO DO: REF: Emsellem et al. 19994]). The density of stellar tracers is assumed to be proportional to the observed and deprojected brightness distribution $\nu(R,z)$ in Eq. (\ref{eq:deprojMGE}). The mass density consists of several sets of MGEs: One MGE, that is usually taken to be $\nu(R,z)$ multiplied by a constant stellar mass-to-light ratio $\Upsilon_*$, describes the distribution of stellar mass in the galaxy. To mimick gradients of stellar mass-to-light ratio, each Gaussian could be assigned its own $\Upsilon_{*,i}$. To add a Navarro-Frenck-White (NFW) [TO DO: REF] dark matter halo component, a MGE generated from a fit to a NFW profile can be added to the stellar component. [TO DO: continue on p. 74] 


\paragraph{Data comparison.} As data we use stellar line-of-sight rotation velocities $v_\text{rot} \equiv \langle v_\text{los} \rangle$ [TO DO: consistent, los or rot] and velocity dispersions $\sigma$ as described in \S\ref{sec:data}. The JAM models give us a prediction for the second line-of-sight velocity moment $v_\text{los}$. The root mean square (rms) line-of-sight velocity $v_\text{rms}$ allows a data-model comparison by relating theses velocities according to 
\begin{equation*}
v_\text{rms}^2 = \langle v_\text{los}^2 \rangle = v_\text{rot}^2 + \sigma^2.
\end{equation*}
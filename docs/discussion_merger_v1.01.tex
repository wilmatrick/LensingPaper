\subsection{Does J1331 have a Merger History?}

\paragraph{TO DO: Stuff to mention}
\begin{itemize}
\item J1331 has a counter-rotating stellar component.
\item suggests that there was a process where two components with angular momenta oriented in opposite directions were involved
\item 1. scenario: accretion of gas on retrograde orbits and subsequent star formation
\item --> counter-rotating component is younger than the rest of the galaxy
\item argument against: To form enough stars such taht the net rotation of the large and massive core is retrograde, a very large amount of gas would have had to be accreted - which is not very likeliy [TO DO" REF: KLEINEBERG 2011] 
\item 2. scenarios: Major mergers of similarly massive galaxies
\item argument against: major merger products are often quiescent, elliptical galaxies [TO DO: REF: TOOMRE TOOMRE 1972]
\item 3. sencaro: minor merger
\item --> dissipationless accretion of a satellite on retrograde orbit [TO DO: REF: KORMENDY 1984] with dense core to survive and spiral to core due to tidal friction
\item --> counter-rotating component would be younger, as star formation in low-mass satellite galaxies occurs later [TO DO: REF]
\item if the counter-rotating component is younger needs to be tested by detailed stellar population analysis
\item Minor merger effects in elliptical galaxies have been studied e.g. by [TO DO:REF: BALCELLS AND QUINN 1990]
\item large bulges of sporals or ellipticals are reddest in the center and become bluer further out
\item Satellite nucleus settling in the core of the larger galaxy could caus a reverse color gradient
\item Normally this gradient is only weak, but there are exceptions [TO DO: CAROLLO 1997].
\item A second effect of the minor merger could be a misalignement of the photometric and kinematic major axis
\end{itemize}

\subsection{Future Work}

Standard JAM modelling approaches seem not to work for J1331. A JAM model for J1331 would need to allow for stellar mass-to-light ratio gradients within the galaxy, velocity anisotropy and a dark matter halo. Because of degeneracies between stellar and dark mass, and matter distribution and anisotropy profile, such a dynamical model would not lead to very tight constraints on the model parameters. Further priors are needed. Detailed stellar population analyses of the spectra taken along J1331's major axis could possibly support or contradict the suspicion of the existence of stellar mass-to-light ratio gradients in J1331.

\paragraph{Model Failures}
\begin{itemize}
\item Assumption of axisymmetry is not true anymore
\item [TO DO: more]
\end{itemize}
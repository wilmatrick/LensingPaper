\subsection{On J1331's possible merger history and potential model failures}

[TO DO: Einheitliche Gross / Klein schreibung in allen Ueberschriften]

\paragraph{TO DO: Stuff to mention}
\begin{itemize}
\item J1331 has a counter-rotating stellar component.
\item suggests that there was a process where two components with angular momenta oriented in opposite directions were involved
\item 1. scenario: accretion of gas on retrograde orbits and subsequent star formation
\item --> counter-rotating component is younger than the rest of the galaxy
\item argument against: To form enough stars such taht the net rotation of the large and massive core is retrograde, a very large amount of gas would have had to be accreted - which is not very likeliy [TO DO" REF: KLEINEBERG 2011] 
\item 2. scenarios: Major mergers of similarly massive galaxies
\item argument against: major merger products are often quiescent, elliptical galaxies [TO DO: REF: TOOMRE TOOMRE 1972]
\item 3. sencaro: minor merger
\item --> dissipationless accretion of a satellite on retrograde orbit [TO DO: REF: KORMENDY 1984] with dense core to survive and spiral to core due to tidal friction
\item --> counter-rotating component would be younger, as star formation in low-mass satellite galaxies occurs later [TO DO: REF]
\item if the counter-rotating component is younger needs to be tested by detailed stellar population analysis
\item Minor merger effects in elliptical galaxies have been studied e.g. by [TO DO:REF: BALCELLS AND QUINN 1990]
\item large bulges of sporals or ellipticals are reddest in the center and become bluer further out
\item Satellite nucleus settling in the core of the larger galaxy could caus a reverse color gradient
\item Normally this gradient is only weak, but there are exceptions [TO DO: CAROLLO 1997].
\item A second effect of the minor merger could be a misalignement of the photometric and kinematic major axis (warp)
\item Assumption of axisymmetry is not true anymore
\item Merger could have changed 3D shape of dark matter halo, or NFW halo was not a good model in the first place (core-cusp problem). However, due to degeneracies in the modelling, we were not able to constrain the profile for a cored logarithmic halo, or to rule it out. 
\end{itemize}

\subsection{Future Work}

Standard JAM modelling approaches seem not to work for J1331. A JAM model for J1331 would need to allow for stellar mass-to-light ratio gradients within the galaxy, velocity anisotropy and a dark matter halo. Because of degeneracies between stellar and dark mass, and matter distribution and anisotropy profile, such a dynamical model would not lead to very tight constraints on the model parameters.  A search for color gradients in J1331 and/or investigation of absorption line indices could support or contradict the suspicion of the existence of stellar mass-to-light ratio gradients in J133. Subsequently detailed stellar population analyses of the spectra taken along J1331's major axis should be conducted to constrain the mass-to-light ratio reliably. In addition, the dynamical modelling should use more of the available information on J1331 and fit dynamics (stellar and gas kinematics from \citet{SWELLSV}) simultaneously with the gravitational lensing (image positions, shape and even flux ratios) in a similar fashion to \citet{SWELLSIV}. To also model the extent, shape and flux of the lensing images, the method by \citet{2004ApJ...611..739T,2003ApJ...590..673W} could be employed, which models the surface brightness distribution of the images and source on a pixelated grid. However, for this to work a good model for the galactic extinction would be needed. High-resolution integral-field spectroscopy could help with this, allow for spatially resolved stellar population analysis and dynamical modelling in two dimensions. This would lead to much better understanding of J1331's structure and mass distribution and therefore answer questions on how minor mergers might modify spiral galaxies.
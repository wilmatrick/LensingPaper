\section{Data} \label{sec:data}

For our lensing and dynamics analysis of J1331 we use on the one hand Hubble Space Telescope (HST) imaging by \cite{SWELLSI} in two filters (F450W and F814W), and on the other hand line-of-sight stellar kinematics along J1331's major axis by \citet{SWELLSV}.

\paragraph{Redshift.} \Wilma{[TO DO: WHO???]} found from SDSS spectra that J1331 has two redshifts inside $1''$: The smaller one, $z_d = 0.113$, is the redshift of J1331 itself, the larger one, $z_s = 0.254$, is the redshift of the lensed background source \citep{SWELLSIII}. According to its redshift and the WMAP5 \Wilma{[TO DO: Do I have to explain the Acronym???]} cosmology \citep{WMAP5cosm}, J1331 has an angular diameter distance of 414 Mpc, which translates into a transverse scaling of $1'' \hat{=} 2.01~\text{kpc}$.

\paragraph{HST imaging.} We use the follow-up HST imaging, with which \citet{SWELLSI} confirmed that J1331 is indeed a strong gravitational lens \Wilma{[TO DO: Was it actually SWELLS I that confirmed that???]}. They performed high resolution imaging with the Hubble Space Telescope's (HST) Wide-Field Planetary Camera 2 (WFPC2) and its WF3 CCD chip.  The images we are using are a combination of four exposures with each an exposure of 400s and were drizzled to a pixel scale of 1 pixel = $0.05''$. In particular we use the images in the filters F450W, to identify the positions of the bluish lensing images, and F814W to create a surface brightness MGE model of the reddish bulge (I-band), used in the JAM dynamical modelling.

\paragraph{Stellar kinematics.} For the dynamical modelling we also use the stellar kinematics for J1331 measured by \citet{SWELLSV}. They obtained long-slit spectra along J1331's major-axis with the Low Resolution Imaging Spectrograph (LRIS) on the Keck I 10m telescope. The width of the slot was $1''$ and the seeing conditions had a FWHM of $\sim 1.1''$. Spectra for spatial bins of different widths along the major axis were extracted. Analogously to \citet{SWELLSII} they measured line-of-sight rotation velocities ($v_\text{rot}$) and stellar velocity dispersion ($\sigma$) by fitting Gaussian line profiles to emission lines in these spectra. Gas kinematics were extracted from fits to H$\alpha$ and NII lines, as tracers for inonized gas.
\\The stellar kinematics, $v_\text{rot}$, $\sigma$ and $v_\text{rms}^2=v_\text{rot}^2 + \sigma^2$ are shown in Figure \ref{fig:kinematics}. The rotation curve reveals a counter-rotating core within $2''\simeq$ 4 kpc. Outside of $\sim 3.5''$ there is a steep drop in the dispersion, which is exptected at the boundary between the pressure supported bulge and the rotationally supported disk, which appears around this radius in the F450W filter in Figure \ref{fig:F450W}. However, in the brighter F814W filter in Figure \ref{fig:F814W}  the large reddish bulge extends out to $\sim5''$. 
\\Inside of $\sim 4''$, the data appears to be symmetric, outside of this the assumption of axisymmetry seems not to be valid anymore, considering the data. We add -2.3 km/s to the $v_\text{rot}$ to ensure $v_\text{rot}(R'=0) \sim 0$ as a possible correction term for a systematic misjudgement of the systemic velocity. We also symmetrize the data within $4''$ and asign a minimum error of $\delta v_\text{rms} > 5$ km/s to the $v_\text{rms}$ data. In the JAM modelling, which is based on the assumption of axisymmetry, only kinematics within $\sim 2.5$ and $4''$ are used. Another reason to restrict to modelling on the bulge region is that our MGE in Table \ref{tab:MGEF814W} is only a good representation of J1331's F814W light distribution inside $\sim 5''$.\\

\Wilma{[TO DO: How to deal with the typo in J1331s name???? SWELLS call it J1331+3638, but its actual coordinates are 1331+3628.]}
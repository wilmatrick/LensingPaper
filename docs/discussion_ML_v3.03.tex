We've presented different dynamical models for the central region of J1331. Some of them capture the observed kinematics, but none of them work at both small and large radii. In the following we discuss possible reasons, also by comparing our results to previous work.

\subsection{On J1331's central stellar mass-to-light ratio} \label{sec:MLdiscussion}

Some of the ambiguities in recovering J1331's matter distribution could be resolved by learning more about stellar populations with different IMFs in J1331. In particular, a sophisticated guess for the stellar mass-to-light ratio in the bulge could be compared to our very reliable measurement of the total mass-to-light ratio inside the Einstein radius $\Upsilon_\text{I,tot}^\text{ein} = 5.56 \Upsilon_{I,\odot}$. This would then either support or contradict the presence of a significant amount of DM in the bulge.

Traditional choices for the IMF are the bottom-heavy IMF by \citet{Salpeter1955},
$$\xi(m) \propto m^{-x}, x=2.35,$$
where $\xi(m) \diff m$ is the number of stars with mass $m$ in $[m,m+\diff m]$, and the IMFs by \citet{2002Sci...295...82K} and \citet{Chabrier2003}, which are in agreement with each other and predict less low-mass stars.

\citet{Ferreras} found a relation between the central stellar velocity dispersion $\sigma_0$ in early-type galaxies and the IMF slope $x$; a higher $\sigma_0$ suggests a more bottom-heavy IMF. For a unimodal (Salpeter-like) IMF and $\sigma_0 \simeq 200~\text{km s}^{-1}$ in J1331 (see Figure \ref{fig:kinematics}) they predict $x \approx 2.33$, which is close to the standard Salpeter slope, also supported by \citet{2014MNRAS.438.1483S}. When assuming a bi-modal (Kroupa-equivalent-like) IMF, \citet{Ferreras} predict $x \approx 2.85$ for J1331's central velocity dispersion. This is more bottom-heavy than the standard \citet{2002Sci...295...82K} IMF. Overall, the central velocity dispersion suggests a rather bottom-heavy IMF in J1331's bulge and therefore large stellar mass-to-light ratio. 

\citet{SWELLSI} estimated J1331's stellar bulge mass given a Salpeter IMF and measured the I-band AB magnitude of the bulge. Transformed to a stellar I-band mass-to-light ratio, their results would correspond to $\Upsilon_\text{I,*}^\text{sal} = 4.7 \pm 1.2$ (see Table \ref{tab:previousresults}). This is not too far from $\Upsilon_\text{I,*} = 4.2 \pm 0.2$ (see Table \ref{tab:modelB4_bestfit}), which we found when including a NFW halo in the JAM modelling.

When \citet{SWELLSI} assumed a Chabrier IMF, their result translates to $\Upsilon_\text{I,*}^\text{chab} = 2.5 \pm 0.6$ (see Table \ref{tab:previousresults}). In Section \ref{sec:results_JAM_SB} we created a dynamical model from only the surface brightness distribution and an increasing mass-to-light ratio profile without additional DM halo. We found that such a model would be perfectly consistent with the Einstein mass, predict a total $\Upsilon_\text{I,tot}(R'\sim0) = 2.53$---being consistent with the Chabrier IMF estimate by \citet{SWELLSI}---and rise quickly to $\Upsilon_\text{I,tot}(R'\gtrsim R_\text{ein}) \gtrsim 6$.

We also compare our results from Section \ref{sec:results_JAM_NFW} with the study by \citet{SWELLSV}. They found that the bulge of J1331 has an IMF \emph{more} bottom-heavy than the Salpeter IMF from fitting a NFW halo to (1) the Einstein mass and (2) gas kinematics at larger radii $\gtrsim 8''$. In Section we fitted a mass model with NFW halo to (1) the Einstein mass and (2) stellar kinematics within $\sim 3.5''$. Our best fit $\Upsilon_\text{I,*} = 4.2 \pm 0.2$ (see Table \ref{tab:modelB4_bestfit}) indicates a \emph{less} bottom-heavy IMF than the Salpeter IMF. \cite{SWELLSV} found systematically lower NFW halo masses ($v_\text{circ,halo}(5'') \sim 120~\text{km s}^{-1}$ according to their Figure 2) than we did ($v_\text{circ,halo}(5'') \sim 200~\text{km s}^{-1}$).

%==============
\begin{table*}
\centering
\begin{tabular}{cccccc}
\hline\hline
& & \multicolumn{2}{c}{Chabrier IMF} & \multicolumn{2}{c}{Salpeter IMF}\\
      &  $L$ [$10^{10}L_{\odot}$]                & $M_*$ [$10^{10}M_\odot$]               & $\Upsilon_\text{I,*}^\text{chab}$ & $M_*$ [$10^{10}M_\odot$] & $\Upsilon_\text{I,*}^\text{sal}$ \\\hline
bulge &   $3.10 \pm 0.15 $  & $7.8 \pm 1.8$ & $2.5 \pm 0.6$ & $14.5 \pm 3.7 $ & $4.7 \pm 1.2$ \\
disk  &   $2.35 \pm 0.11 $  & $2.9 \pm 0.7$ & $1.2 \pm 0.3$ & $5.2 \pm 1.1$ & $2.2 \pm 0.5$ \\
total &   $5.45 \pm 0.19$ & $10.6 \pm 1.9$& & $19.7 \pm 3.9$&\\\hline
\end{tabular}
\caption{Total I-band luminosity, stellar mass and mass-to-light ratio, calculated from the I-band AB magnitudes and stellar masses found for J133's bulge and disk by \citet{SWELLSI} (their table 2) for comparison with this work. The transformation from AB magnitudes to the Johnson-Cousins I-Band used the relation $I[\text{mag}] = I[\text{ABmag}] - 0.309$ from \citet{FG1994} (their table 2). For the conversion from apparent magnitude to total luminosity the redshift $z=0.113$ \citet{SWELLSIII} was turned into a luminosity distance using the cosmology by \citet{WMAP5cosm}. }
\label{tab:previousresults}
\end{table*}
%==============